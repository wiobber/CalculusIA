\documentclass{ximera}

\input{../preamble.tex}


\outcome{Take derivatives of compositions of functions using the chain rule.}
\outcome{Take derivatives that require the use of multiple derivative rules.}
\outcome{Use the chain rule to calculate derivatives from a table of values.}
\outcome{Use order of operations in situations requiring multiple derivative rules.}


\title{3.3 - The Chain Rule}

\begin{document}
\begin{abstract}
  Here we compute derivatives of compositions of functions
\end{abstract}
\maketitle

So far we have seen how to compute the derivative of a function built up from other functions by addition, subtraction, multiplication and
division. There is another very important way that we combine
functions: composition. The \textit{chain rule} allows us to deal with
this case. Consider
\[
h(x) = (3x+1)^2.
\] 
Notice, we can rewrite $h(x)$ as $h(x)=9x^2+6x+1$ and then apply our basic derivative rules to find that $h'(x)=18x+6=6(3x+1)$. But, what if we have something like 
\[h(x)=(2x+1)^{100}?\]
In this case, we don't want to multiply this all out and then find the derivative using our basic derivative rules because who has time for all that.
If we let $f(x) = x^{100}$ be our outside function and $g(x) = 2x+1$ be our inside function, then we can express
$h(x) = f(g(x))$. The question is, can we compute the derivative of a
composition of functions using the derivatives of the inner and outer functions
$f(x)$ and $g(x)$? To do so, we need the \textit{chain rule}.



\begin{theorem}[Chain Rule]\index{chain rule}\index{derivative rules!chain}
If $f$ and $g$ are differentiable, then
\[
\ddx f(g(x)) = f'(g(x))g'(x).
\]
\end{theorem}



It will take a bit of practice to make the use of the chain rule come
naturally, it is more complicated than the earlier differentiation
rules we have seen. Let's return to our motivating example.

\begin{example}
Compute:
\[
\ddx (2x+1)^{100}
\]

\begin{explanation}
Set $f(x) = x^{100}$ and $g(x) = 2x+1$, now
\[
f'(x) = \answer[given]{100x^{99}} \qquad\text{and}\qquad g'(x) = \answer[given]{2}.
\]
Hence
\begin{align*}
\ddx (2x+1)^{100} &= \ddx f(g(x))\\ 
&=f'(g(x))g'(x) \\
&= 100(\answer[given]{2x+1})^{99}\cdot \answer[given]{2}\\
&= 200(2x+1)^{99}.
\end{align*}
\end{explanation}
\end{example}

Let's return to our very first example where $h(x)=(3x+1)^2$ and we found that by expanding and applying basic derivative rules, we could get $h'(x)=6(3x+1)$. Let's confirm that we get the same result when using the chain rule.

\begin{example}
    Compute:
    \[\frac{d}{dx}(3x+1)^2\]
    \begin{explanation}
        We know $f(x) = x^{2}$ is our outside function and $g(x) = 3x+1$ is our inside function. Thus 
        \[f'(x)=\answer[given]{}\]
    \end{explanation}
\end{example}

Let's see a more complicated chain of compositions.

\begin{example}
Compute:
\[
\ddx \sqrt{1+\sqrt{x}}
\]

\begin{explanation}
Set 
$f(x)=\sqrt{x}$ and $g(x)=1+x$. Hence,
\[
\sqrt{1+\sqrt{x}}=f(g(\answer[given]{f}(x)))
\]
and by the chain rule we know
\[
\ddx f(g(f(x))) = f'(g(f(x)))g'(f(x))f'(x).
\]
Since 
\[
f'(x) = \answer[given]{\frac{1}{2\sqrt{x}}} \qquad\text{and}\qquad g'(x) = \answer[given]{1}
\]
We have that
\[
\ddx \sqrt{1+\sqrt{x}} = \frac{1}{2\sqrt{1+\sqrt{x}}}\cdot 1\cdot  \answer[given]{\frac{1}{2\sqrt{x}}}.
\]
\end{explanation}
\end{example}

The chain rule allows to differentiate compositions of functions that
would otherwise be difficult to get our hands on.

\begin{example}
Compute:
\[
\ddx \frac{1}{\sqrt[3]{(3x^6+4x^2-1)^{9}-34}}
\]

\begin{explanation}
Set $f(x) = \answer[given]{x^{-1/3}}$, $g(x) = \answer[given]{x^9-34}$, and $h(x) = \answer[given]{3x^6+4x^2-1}$
so that $f(g(h(x))) = \frac{1}{\sqrt[3]{(3x^6+4x^2-1)^{9}-34}}$. Now
\begin{align*}
  \ddx \frac{1}{\sqrt[3]{(3x^6+4x^2-1)^{9}-34}} &= \ddx f(g(h(x)))\\
  &= f'(g(h(x))) \cdot g'(h(x)) \cdot h'(x)\\
  &= -\frac{1}{3}({\answer[given]{(3x^6+4x^2-1)^{9}-34}})^{-4/3} \cdot 9(\answer[given]{3x^6+4x^2-1})^{8} \cdot (\answer[given]{18x+8x}).
\end{align*}
\end{explanation}
\end{example}



Depending on your feelings about the quotient rule, with the chain rule, you now have the opportunity to bypass it entirely in favor of using a combination of the chain rule and the product rule.

\begin{example}
Compute
\[
\ddx \frac{x^3}{x^2+1}
\]
without using the quotient rule.
\begin{explanation}
Rewriting this as 
\[
\ddx x^3(x^2+1)^{-1}, 
\]
set $f(x) = \answer[given]{x^{-1}}$ and $g(x) = \answer[given]{x^2+1}$ so that $f(g(x)) = (x^2 + 1)^{-1}$. Now
\[
x^3(x^2+1)^{-1} = x^3 f(g(x)),
\]
and by the product and chain rules
\[
\ddx x^3 f(g(x)) = \answer[given]{3x^2} \cdot f(g(x))+ \answer[given]{x^3} \cdot f'(g(x))g'(x).
\]
Since $f'(x) = \answer[given]{\frac{-1}{x^2}}$ and $g'(x) = \answer[given]{2x}$, write
\[
\ddx \frac{x^3}{x^2+1} = \frac{3x^2}{x^2+1}-\frac{2x^4}{(x^2+1)^2}.
\]
\end{explanation}
\end{example}
\begin{example}
    Suppose $f$ is a function whose values are given in the following table.
    \begin{center}
        \begin{tabular}{c|cc}
        \hline
             $x$& $f(x)$ &$f'(x)$\\
             \hline
            1 & -3 &4\\
             2&  5&-1\\
             3&  2&0\\
        \end{tabular}
    \end{center}
    Find $\left[\ddx ((4x-3)f(x^2+1))\right]_{x=1}$
    \begin{explanation}
        We can see that the first operation to pay attention is to multiplication. So the product rule is the first rule to apply.
        \begin{align*}
        \ddx((4x-3)f(x^2+))&=(\answer[given]{4})\cdot f(x^2+1)+(4x-3)\cdot \ddx(f(x^2+1))\\
        &=4\cdot f(x^2+1)+(4x-3)f'(x^2+1)\cdot (\answer[given]{2x})
        \end{align*}
        Evaluating our derivative at $x=1$ gives:
        \begin{align*}
            \left[\ddx ((4x-3)f(x^2+1))\right]_{x=1}&=4f(2)+(1)f'(2)\cdot 2\\
            &=4\cdot(\answer[given]{5})+2\cdot(\answer[given]{-1})\\
            &=\answer[given]{18}
        \end{align*}
    \end{explanation}
\end{example}




\end{document}