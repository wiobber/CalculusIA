\documentclass{ximera}

\input{../preamble.tex}

\outcome{To be able to use the method of substitution to solve some ``simple'' integrals, with an emphasis on being able to correctly identify what to substitute for.}
\outcome{Undo the Chain Rule.}
\outcome{Calculate indefinite integrals (antiderivatives) using basic substitution.}
\outcome{Calculate definite integrals using basic substitution.}
\title{8.10 - The Substitution Rule}
\begin{document}
\begin{abstract}
  We learn a new technique, called substitution, to help us solve
  problems involving integration.
\end{abstract}
\maketitle


Computing antiderivatives is not as easy as computing derivatives. For instance, we've seen that 
\[\int f(x)g(x)\d x \neq \int f(x)dx\int g(x)\d x.
\]
Additionally, the chain rule can be difficult to ``undo.''  We
have a general method called ``integration by substitution'' that will
somewhat help with this difficulty. 

If the functions are differentiable, we can apply the chain rule and obtain
\[
\ddx f(g(x)) = f'(g(x))g'(x)
\]
If the derivatives are continuous, we can use this equality to evaluate a definite integral. Namely,

 \[ \int f'(g(x))g'(x) \d x = {f(g(x))}+C \]

Let's start with a simple example that fits the above form.
\begin{example}
Compute:
\[
\int 2x e^{x^2}\d x
\]
\begin{explanation}
Notice that $\frac{d}{dx}\left[e^{x^2}\right]=e^{x^2}\frac{d}{dx}\left[x^2\right]=2xe^{x^2}$.

Thus, we know that 
\[
\int 2x e^{x^2}\d x=e^{x^2}+C.
\]
However, there are many integrals that we encounter that are not so obviously in that form. So how do we deal with those scenarios?

Notice this integral has the structure
\[
\int f'(g(x)) g'(x) \d x ,  
\]


where $g(x) =x^2$. Thus $g'(x)dx =2xdx$, and note that

\[
\int 2x e^{x^2}\d x
=\int e^{\underbrace{x^2}}\underbrace{2x\d x}
\]
So, let's introduce a new variable $u$. We will let $u=g(x)=x^2$. We now want to substitute anything that has an $x$ in it with it's corresponding expression in terms of $u$.
If 
\begin{align*}
    u&=x^2\\
    du&=2xdx
\end{align*}

Notice that the second line can be seen as using differentials.


\[
\int 2x e^{x^2}\d x=\int e^{g(x)}g'(x)\d x=\int e^{u}\d u
\]

We can easily evaluate the last integral
\[
\int e^{u}\d u= e^u+C
\]

We now need to substitute the $u$ in our answer with it's equivalent expression in terms of $x$. 
Thus,

\[
\int 2x e^{x^2}\d x=\int e^{u}\d u=e^u+C=e^{x^2}+C.
\]

Therefore, whenever we are faced with a problem of evaluating a difficult integral, we try to  replace it with  a simpler one.


\end{explanation}
\end{example}


\section{More examples}

With some experience, it is (usually) not too hard to see what to
substitute as $u$.  When determining what your $u$ should be, it is often helpful to look for an expression whose derivative (which can differ by a constant) is multiplied into the function. We will work through the following examples in the
same way that we did above.
\begin{example}
Compute:
\[
\int x^4(x^5+1)^{9} \d x
\]
\begin{explanation}
Here we set $u =  \answer[given]{x^5+1}$, so $\d u =  \answer[given]{5x^4} \d x$.  Then
we can solve for the $x^4$ that is present in the integrand. It follows that $\frac{1}{5}\d u=x^4\d x$. Therefore
\begin{align*}
  \int x^4(x^5+1)^{9} \d x &= \int \frac{1}{5} (u)^{9} \d u \\
  &= \frac{1}{5} \int u^{9} \d u\\
&=\frac{u^{10}}{ \answer[given]{50}}+C.
\end{align*}
We again need to back-substitute into our answer, so that our final
answer is a function of $x$.  Recalling that $u= x^5+1$, we have
our final answer
\[
\int x^4(x^5+1)^{9} \d x= \frac{(x^5+1)^{10}}{\answer[given]{50}}+C.
\]
Reminder: you can always verify your result by differentiating.

\end{explanation}
\end{example}


%If substitution works to solve an integral (and that is not always the
%case!), a common trick to find what to substitute for is to locate the
%``ugly'' part of the function being integrated.  We then substitute
%for the ``inside'' of this ugly part.  While this technique is
%certainly not rigorous, it can prove to be very helpful.  This is
%especially true for students new to the technique of substitution.
%The next two problems are really good examples of this philosophy.


\begin{example}
  Compute:
  \[
  \int \frac{\cos(\ln x)}{x} \d x
  \]
\begin{explanation}
Here the ``ugly'' part here is $\cos(\ln x)$.  So we substitute for
the inside:
\[
u=g(x)=\answer[given]{\ln x}.
\]
Then
\[
\d u =  \answer[given]{\frac{1}{x}} \d x 	
\]

Then we substitute into the original integral and solve:
\begin{align*}
\int\frac{\cos(\ln x)}{x} \d x &= \int\frac{\cos(u)}{x} x \d u  \\
&= \int \cos(u) \d u  \\
&= {\answer[given]{\sin(u)}}+C\\
&= \answer[given]{\sin}(\answer[given]{\ln(x)})+C
\end{align*}
\end{explanation}
\end{example}
Let's look at another example that will require a new technique.
\begin{example}
  Compute:
\[
\int x^2\sqrt{1-x}\d x
\]
\begin{explanation}

Here it is not apparent that the chain rule is involved. However, if
it was involved, perhaps a good guess for $g$ would be
\[
u = \answer[given]{1-x}
\]
and then
\begin{align*}
  \d u &= \answer[given]{-1} \d x, \\
  -\d u &= \d x.
\end{align*}
Now we consider the integral we are trying to compute
\[
\int x^2\sqrt{1-x}\d x
\]
and we substitute using our work above. Write with me
\begin{align*}
  \int x^2\sqrt{1-x}\d x &= \int x^2 \answer[given]{\sqrt{u}} ( \answer[given]{-1} ) \d u \\
  &= \int {-x^2 \answer[given]{\sqrt{u}}}\d u.
\end{align*}
At this point, it seems like this isn't going to work since we do not know how to substitute $x^2$ with an expression in terms of $u$.

Let's dig into this a little bit. We know $u=1-x$. With a little rearranging perhaps we can find something helpful!
\begin{align*}
u &= 1-x \\
 u -1 &= -x\\
 \answer[given]{1- u} &= x
\end{align*}
so now we can replace the $x^2$ as well. Notice 
\begin{align*}
    x^2&=(1-u)^2\\
    &=(1-2u+u^2)
\end{align*}
Picking up where we left off, 
\[ \int x^2\sqrt{1-x}\d x = \int -\answer[given]{(1-2u+u^2)} \sqrt{u}\d u.
\]
At this point, we are close to being done. Write
\begin{align*}
\int -{(1-2u+u^2)} \sqrt{u}\d u &= \int -u^{1/2}{(1-2u+u^2)} \d u\\
&=\int -u^{1/2}+2u^{3/2}-u^{5/2}\d u\\
&=-\answer[given]{\frac{2}{3}}u^{3/2}+\frac{4}{5}u^{\answer[given]{5/2}}-\frac{2}{7}u^{7/2}+C
\end{align*}
Now recall that $u = 1-x$. Hence our final answer is
\[
\int x^3\sqrt{1-x}\d x = -\answer[given]{\frac{2}{3}}(1-x)^{3/2}+\frac{4}{5}(1-x)^{\answer[given]{5/2}}-\frac{2}{7}(1-x)^{7/2}+C
\]
\end{explanation}
\end{example}



\begin{example}\label{key example}
Compute:
\[
\int \frac{u}{1-u^2} \d u
\]
\begin{explanation}
We want to substitute for $1-u^2$.  
But the variable ``$u$'' has already been used. We can substitute with whatever variable that we want.  
In particular, let's use ``$w$'' for this problem.  
So we let
\[
w = 1 - u^2
\]
and then
\begin{align*}
  \d w &= \answer[given]{-2u} \d u,\\
  \answer[given]{-\frac{1}{2}}\d w &=u  \d u .
\end{align*}
Thus
\begin{align*}
\int \frac{u}{1-u^2} \d u &= \int \left(-\frac{1}{2}\right)\left(\frac{1}{w}\right)\d w  \\
&= - \frac{1}{2} \int \frac{1}{w} \d w  \\
&= - \frac{1}{2} \ln|w| + C  \\
&= - \frac{1}{2} \ln|\answer[given]{1-u^2}| + C.
\end{align*}
\end{explanation}


\end{example}

There are many functions that may not look like prime candidates for integrating using substitution, but with a little rewriting it becomes clear that they are! One such example is $y=\tan (x)$

\begin{example}\label{example tan}
Compute:
\[
\int \tan(x) \d x
\]
\begin{explanation}
We begin by recalling that 

$\tan(x) = \frac{\sin(x)}{\cos(x)}  $
We then make the substitution
\[
u = \cos(x)
\]
and so
\begin{align*}
\d u &= \answer[given]{-\sin(x)} \d x,\\
-\d u &= {\answer[given]{\cos(x)}} \d x.
\end{align*}
Then
\begin{align*}
\int \tan(x) \d x &= \int \frac{\sin(x)}{\cos(x)} \d x  \\
&= \int -\frac{1}{u}  \d u  \\
&= \answer[given]{-\ln|u|}+C
\end{align*}
Substituting with $u=\cos(x)$, we get
\[\int \tan(x) \d x=-\ln|\answer[given]{\cos(x)}|+C=\ln|\sec(x)|+C\]
Notice that last equality was achieved using the log rule
\[n\log(x)=\log(x^n)\]
\end{explanation}
\end{example}

We have just proved

\begin{theorem}
\[
\int \tan(x) \d x = \ln|\sec(x)| + C.
\]
\end{theorem}


To summarize, if we suspect that a given function is the derivative of
another via the chain rule, we introduce a new variable $u=g(x)$, where $g$ is a likely candidate for
the inner function. We rewrite the integral
 entirely in terms of $u$, with no $x$ remaining in the
expression. If we can integrate this new function of $u$, then the
antiderivative of the original function is obtained by replacing $u$
by $g(x)$.
\end{document}