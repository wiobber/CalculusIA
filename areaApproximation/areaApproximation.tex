\documentclass{ximera}

\input{../preamble.tex}

\title{8.2 - Approximating area with rectangles}

\outcome{Define area.}
\outcome{Understand the relationship between area under a curve and sums of rectangles.}
\outcome{Approximate area under a curve.}
\outcome{Compute left, right, and midpoint Riemann sums with 10 or fewer rectangles.}

\begin{document}
\begin{abstract}
  We introduce the basic idea of using rectangles to approximate the
  area under a curve.
\end{abstract}
\maketitle

\section{Rectangles and areas}
We can calculate areas of many different shapes: rectangles, triangles, trapezoids, circles, ... etc.
What about the area of something like the region under the graph of a function?

We want to compute the area between the curve $y=f(x)$ and the
horizontal axis on the interval $[a,b]$:
\begin{image}
  \begin{tikzpicture}[
      declare function = {f(\x) = -sin(deg(\x)) + 3;} ]
	\begin{axis}[
            domain=-.2:7, xmin =-.2,xmax=7,ymax=5,ymin=-.2,
            width=6in,
            height=3in,
            xtick={1,6}, 
            xticklabels={$a$,$b$},
            ytick style={draw=none},
            yticklabels={},
            axis lines=center, xlabel=$x$, ylabel=$y$,
            every axis y label/.style={at=(current axis.above origin),anchor=south},
            every axis x label/.style={at=(current axis.right of origin),anchor=west},
            axis on top,
          ]
          \addplot [draw=none,fill=fillp,domain=1:6, smooth] {f(x)} \closedcycle;
          \addplot [very thick,penColor, smooth] {f(x)};
        \end{axis}
\end{tikzpicture}
\end{image}
One way to do this would be to approximate the area with
rectangles. With one rectangle we get a rough approximation:
\begin{image}
  \begin{tikzpicture}[
      declare function = {f(\x) = -sin(deg(\x)) + 3;}]
    \begin{axis}[  
        domain=-.2:7, xmin =-.2,xmax=7,ymax=5,ymin=-.7,
        width=6in,
        height=3in,xtick={1,6},
        xticklabels={$a$,$b$},
        ytick style={draw=none},
        yticklabels={},
        axis lines=center, xlabel=$x$, ylabel=$y$,
        every axis y label/.style={at=(current axis.above origin),anchor=south},
        every axis x label/.style={at=(current axis.right of origin),anchor=west},
        axis on top,
      ]
      \foreach \rectnumber in {1}
               {
                 \addplot [draw=penColor,fill=fillp] plot coordinates
                          {({1+(\rectnumber - 1) * 5},{f(1+\rectnumber * 5)})
                            ({1+(\rectnumber) * 5},{f(1+\rectnumber * 5) })} \closedcycle;
               };
               \addplot [very thick,penColor, smooth] {f(x)};
               \node[fillp!50!black] at (axis cs:3.5,1) {\huge$\mathsf{1}$};
    \end{axis}
  \end{tikzpicture}
\end{image}
Two rectangles might make a better approximation:
\begin{image}
  \begin{tikzpicture}[
      declare function = {f(\x) = -sin(deg(\x)) + 3;}]
    \begin{axis}[  
        domain=-.2:7, xmin =-.2,xmax=7,ymax=5,ymin=-.7,
        width=6in,
        height=3in,xtick={1,6},
        xticklabels={$a$,$b$},
        ytick style={draw=none},
        yticklabels={},
        axis lines=center, xlabel=$x$, ylabel=$y$,
        every axis y label/.style={at=(current axis.above origin),anchor=south},
        every axis x label/.style={at=(current axis.right of origin),anchor=west},
        axis on top,
      ]
      \foreach \rectnumber in {1,2}
               {
                 \addplot [draw=penColor,fill=fillp] plot coordinates
                          {({1+(\rectnumber - 1) * 5/2},{f(1+(\rectnumber-1) * 5/2)})
                            ({1+(\rectnumber) * 5/2},{f(1+(\rectnumber-1) * 5/2) })} \closedcycle;
               };
               \addplot [very thick,penColor, smooth] {f(x)};
               \node[fillp!50!black] at (axis cs:2.25,1) {\huge$\mathsf{1}$};
               \node[fillp!50!black] at (axis cs:4.75,1) {\huge$\mathsf{2}$};
    \end{axis}
  \end{tikzpicture}
\end{image}
With even more, we get an even closer approximation:
\begin{image}
  \begin{tikzpicture}[
      declare function = {f(\x) = -sin(deg(\x)) + 3;}]
    \begin{axis}[  
        domain=-.2:7, xmin =-.2,xmax=7,ymax=5,ymin=-.7,
        width=6in,
        height=3in,xtick={1,6},
        xticklabels={$a$,$b$},
        ytick style={draw=none},
        yticklabels={},
        axis lines=center, xlabel=$x$, ylabel=$y$,
        every axis y label/.style={at=(current axis.above origin),anchor=south},
        every axis x label/.style={at=(current axis.right of origin),anchor=west},
        axis on top,
      ]
      \foreach \rectnumber in {1,2,...,5}
               {
                 \addplot [draw=penColor,fill=fillp] plot coordinates
                          {({1+(\rectnumber - 1) * 5/5},{f(1+\rectnumber * 5/5)})
                            ({1+(\rectnumber) * 5/5},{f(1+\rectnumber * 5/5) })} \closedcycle;
               };
               \addplot [very thick,penColor, smooth] {f(x)};
               \node[fillp!50!black] at (axis cs:1.5,1) {\huge$\mathsf{1}$};
               \node[fillp!50!black] at (axis cs:2.5,1) {\huge$\mathsf{2}$};
               \node[fillp!50!black] at (axis cs:3.5,1) {\huge$\mathsf{3}$};
               \node[fillp!50!black] at (axis cs:4.5,1) {\huge$\mathsf{4}$};
               \node[fillp!50!black] at (axis cs:5.5,1) {\huge$\mathsf{5}$};       
    \end{axis}
  \end{tikzpicture}
\end{image}





\begin{definition}
  If we are approximating the area between a curve and the $x$-axis
  on $[a,b]$ with $n$ rectangles of width $\Delta x$, then
  \[
  \Delta x = \frac{b-a}{n}.
  \]
\end{definition}
\begin{image}
  \begin{tikzpicture}[
      declare function = {f(\x) = -sin(deg(\x)) + 3;}]
    \begin{axis}[  
        domain=-.2:7, xmin =-.2,xmax=7,ymax=5,ymin=-.7,
        width=6in,
        height=3in,xtick={1,6},
        xticklabels={$a$,$b$},
        ytick style={draw=none},
        yticklabels={},
        axis lines=center, xlabel=$x$, ylabel=$y$,
        every axis y label/.style={at=(current axis.above origin),anchor=south},
        every axis x label/.style={at=(current axis.right of origin),anchor=west},
        axis on top,
      ]
      \foreach \rectnumber in {1,2,...,5}
               {
                 \addplot [draw=penColor,fill=fillp] plot coordinates
                          {({1+(\rectnumber - 1) * 1},{f(1+\rectnumber * 1)})
                            ({1+(\rectnumber) * 1},{f(1+\rectnumber * 1) })} \closedcycle;

                 \addplot[decoration={brace,mirror,raise=.2cm},decorate,thin] plot coordinates
                          {(\rectnumber+.1,0)
                            (1+\rectnumber-.1,0)};
               };
               \addplot [very thick,penColor, smooth] {f(x)};
               \node at (axis cs:1.5,-.5) {$\Delta x$};
               \node at (axis cs:2.5,-.5) {$\Delta x$};
               \node at (axis cs:3.5,-.5) {$\Delta x$};
               \node at (axis cs:4.5,-.5) {$\Delta x$};
               \node at (axis cs:5.5,-.5) {$\Delta x$};
    \end{axis}
  \end{tikzpicture}
\end{image}
\begin{question}
  Suppose we wanted to approximate area between the curve $y=x^2+1$
  and the $x$-axis on the interval $[-1,1]$, with $8$ rectangles. What is $\Delta x$?
  \begin{prompt}
    \[
    \Delta x = \answer{1/4}
    \]
  \end{prompt}
\end{question}

As we add rectangles, we are more closely approximating the area we are interested in:
\begin{image}
  \begin{tikzpicture}[
      declare function = {f(\x) = -sin(deg(\x)) + 3;}]
    \begin{axis}[  
        domain=-.2:7, xmin =-.2,xmax=7,ymax=5,ymin=-.7,
        width=6in,
        height=3in,xtick={1,6},
        xticklabels={$a$,$b$},
        ytick style={draw=none},
        yticklabels={},
        axis lines=center, xlabel=$x$, ylabel=$y$,
        every axis y label/.style={at=(current axis.above origin),anchor=south},
        every axis x label/.style={at=(current axis.right of origin),anchor=west},
        axis on top,
      ]
      \foreach \rectnumber in {1,2,...,30}
               {
                 \addplot [draw=penColor,fill=fillp] plot coordinates
                          {({1+(\rectnumber - 1) * 5/30},{f(1+\rectnumber * 5/30)})
                            ({1+(\rectnumber) * 5/30},{f(1+\rectnumber * 5/30) })} \closedcycle;
               };
               \addplot [very thick,penColor, smooth] {f(x)};
    \end{axis}
  \end{tikzpicture}
\end{image}
%We could find the area exactly if we could compute the limit as the
%width of the rectangles goes to zero and the number of rectangles goes
%to infinity.


Let's setup some notation to help with these calculations:

\begin{definition}

Notice that $\Delta x$ tells us the length of the base of each rectangle. We can also think about it as the interval from $x=a=x_0$ to $x=b=x_n$ gets split into smaller \emph{subintervals} of length $\Delta x$. So, we now have subintervals $[x_0,x_1],[x_1,x_2],\dots, [x_{n-1},x_n]$. Notice, the subinterval $[x_{k-1},x_k]$ forms the base of the $k$th rectangle. 

  \end{definition}
  
  In the graph below, we've numbered the rectangles to
  help you see the relation between the subintervals
  and the $k$th rectangle.
\begin{image}
  \begin{tikzpicture}[
      declare function = {f(\x) = -sin(deg(\x)) + 3;}]
    \begin{axis}[  
        domain=-.2:7, xmin =-.2,xmax=7,ymax=5,ymin=-.7,
        width=6in,
        height=3in,xtick={1,2,...,6},
        xticklabels={$x_0=a$,$x_1$,$x_2$,$x_3$,$x_4$,$x_5=b$},
        ytick style={draw=none},
        yticklabels={},
        axis lines=center, xlabel=$x$, ylabel=$y$,
        every axis y label/.style={at=(current axis.above origin),anchor=south},
        every axis x label/.style={at=(current axis.right of origin),anchor=west},
        axis on top,
      ]
      \foreach \rectnumber in {1,2,...,5}
               {
                 \addplot [draw=penColor,fill=fillp] plot coordinates
                          {({1+(\rectnumber - 1) * 5/5},{f(1+\rectnumber * 5/5)})
                            ({1+(\rectnumber) * 5/5},{f(1+\rectnumber * 5/5) })} \closedcycle;
               };
               \addplot [very thick,penColor, smooth] {f(x)};
               \node[fillp!50!black] at (axis cs:1.5,1) {\huge$\mathsf{1}$};
               \node[fillp!50!black] at (axis cs:2.5,1) {\huge$\mathsf{2}$};
               \node[fillp!50!black] at (axis cs:3.5,1) {\huge$\mathsf{3}$};
               \node[fillp!50!black] at (axis cs:4.5,1) {\huge$\mathsf{4}$};
               \node[fillp!50!black] at (axis cs:5.5,1) {\huge$\mathsf{5}$};       
    \end{axis}
  \end{tikzpicture}
\end{image}
Note, if we are approximating the area between a curve and the
horizontal axis on $[a,b]$ with $n$ rectangles, then it is always the
case that
\[
x_0=a\qquad\text{and}\qquad x_n = b.
\]


\begin{question}
  If we are approximating the area between a curve and the horizontal
  axis with $11$ rectangles, what will be the subinterval for the 9th rectangle?
  \begin{hint}
    You can draw it!
  \end{hint}
  \begin{prompt}
    The subinterval will be $[x_8,x_9]$
  \end{prompt}
\end{question}







\section{But which set of rectangles?}

When we use $n$ rectangles to compute the area under a curve, \textit{the width} of each rectangle is given by $\Delta x=\frac{b-a}{n}$. 
It is clear that $\Delta x=x_k-x_{k-1}$, for $k=1,\dots, n$. 

But how do we determine  \textit{the height} of the rectangle?

One method is to choose the left endpoint of the subinterval that forms the base of the rectangle and determine the height of the rectangle based on that.

\begin{example}
    Consider the function $f(x)=x^2+2$ over the interval $[0,4]$. Approximate the area between $f$ and the $x$-axis between $x=0$ and $x=4$ using 4 rectangles and heights based on left endpoints.

    \begin{explanation}
        Lets start with an image of this situation.

        \begin{image}
  \begin{tikzpicture}[
      declare function = {f(\x) = pow(\x,2) +2;}
      ]
	\begin{axis}[
            domain=0:4, xmin =-1,xmax=5,ymax=18,ymin=-1,
            width=6in,
            height=3in,
            axis lines=center, xlabel=$x$, ylabel=$y$,
            every axis y label/.style={at=(current axis.above origin),anchor=south},
            every axis x label/.style={at=(current axis.right of origin),anchor=west},
            xtick={0,1,2,3,4},
            xticklabels={$0$,$1$,$2$,$3$,$4$},
            ytick style={draw=none},
            yticklabels={},
            axis on top,
          ]
          \foreach \rectnumber in {1,2,...,4}
                   {
                     \addplot [draw=penColor,fill=fillp] plot coordinates
                              {({(\rectnumber-1) },{f((\rectnumber-1) )})
                                ({(\rectnumber)},{f((\rectnumber-1) ) })} \closedcycle;
               };
                   \addplot [very thick,penColor, smooth] {f(x)};
                   \node[fillp!50!black] at (axis cs:0.5,.9) {\huge$\mathsf{1}$};
               \node[fillp!50!black] at (axis cs:1.5,1.2) {\huge$\mathsf{2}$};
               \node[fillp!50!black] at (axis cs:2.5,2.4) {\huge$\mathsf{3}$};
               \node[fillp!50!black] at (axis cs:3.5,4.6) {\huge$\mathsf{4}$};
               
        \end{axis}
\end{tikzpicture}
\end{image}

We can see each of our 4 rectangles and also see that the height of each rectangle is determined based on the function value at the left endpoint of each sample point. 

First, let's determine the width of each rectangle.
\[\Delta x=\frac{b-a}{n}=\frac{\answer{4}-0}{\answer{4}}=\answer{1}\]

Let's determine the height of each rectangle by plugging in the left endpoint of the subinterval into our function $f(x)=x^2+2$. We can compile this information in a table.

\[
\begin{array}{c|c|c}
\text{index}&\text{left endpoint}&\text{rectangle height}\\
k&x_{k-1} & f(x_{k-1}) \\ \hline
1&\answer[given]{0}  &  (0)^2+2=2 \\
2&\answer[given]{1}     & \answer[given]{3}   \\
3&\answer[given]{2}       & \answer[given]{6}   \\
4&\answer[given]{3}     & \answer[given]{11}   
\end{array}
\]

We can now calculate our approximation of the area between $f(x)$ and the $x$-axis between $x=0$ and $x=4$ by adding together the area of each rectangle.

Thus,
\begin{align*}
Area &\approx f(0)\Delta x+f(1)\Delta x+f(2)\Delta x+f(3) \Delta x \\
&= \answer[given]{2}(1)+\answer[given]{3}(1)+\answer[given]{6}(1)+\answer[given]{11}(1)\\
&= 22
\end{align*}
    \end{explanation}
\end{example}

In the previous example, we picked the left endpoint of each subinterval and plugged it into our function to determine the height of the rectangle. Is that the only way to do it? Let's investigate this more generally.

  Instead of specifying the point to be the left endpoint of each subinterval, generally, we choose a \textit{sample point} $x_k^*$ that lies in the subinterval $[x_{k-1},x_k] $ and evaluate the function at that point. The value $f(x_k^*)$ determines the height of a rectangle.
\begin{image}
  \begin{tikzpicture}[
      declare function = {f(\x) = -sin(deg(\x)) + 3;}]
    \begin{axis}[  
        domain=-.2:7, xmin =-.2,xmax=7,ymax=5,ymin=-.7,
        width=6in,
        height=3in,xtick={1,3,3.4,4,6},
        xticklabels={$x_0=a$,$x_{k-1}$,$x^*_{k}$,$x_k$,$x_n=b$},
        ytick style={draw=none},
        yticklabels={},
        axis lines=center, xlabel=$x$, ylabel=$y$,
        every axis y label/.style={at=(current axis.above origin),anchor=south},
        every axis x label/.style={at=(current axis.right of origin),anchor=west},
        axis on top,
      ]
      \foreach \rectnumber in {3}
               {
                 \addplot [draw=penColor,fill=fillp] plot coordinates
                          {({1+(\rectnumber - 1) * 5/5},{f(0.4+\rectnumber * 5/5)})
                            ({1+(\rectnumber) * 5/5},{f(0.4+\rectnumber * 5/5) })} \closedcycle;
               };
               \addplot [ultra thick,penColor, smooth] {f(x)};
                  \addplot [ultra thick,penColor2, smooth] plot coordinates {(3.4,0) (3.4,3.28)};
               \node[fillp!50!black] at (axis cs:3.8,1.5) {\huge$\mathsf{k}$};
                 \node[fillp!50!black] at (axis cs:2.5,1.5) {${f(x^*_{k})}$};
                  \node[fillp!50!black] at (axis cs:3.5,3.8) {${\Delta x}$};
                   \node[fillp!50!black] at (axis cs:1.5,3.8) {\huge${y=f(x)}$};
                     \addplot[decoration={brace,raise=.2cm},decorate,thin] plot coordinates
                       {(2.95,.1) (2.95,{f(3.4)-.1})};
               \addplot[decoration={brace,raise=.2cm},decorate,thin] plot coordinates
                       {(3,3.35) (4,3.35)};

    \end{axis}
  \end{tikzpicture}
\end{image}

\begin{definition}
  When approximating an area with rectangles, a \dfn{sample point} is
  the $x$-coordinate in the subinterval $[x_{k-1},x_k]$ that determines the height of $k^{th}$
  rectangle. For  $k=1,\dots, n$, we denote a sample point as:
  \[
  x_k^*
  \]
  and the value
   \[
 f( x_k^*)
  \]
  is the height of the $k^{th}$ rectangle.
\end{definition}
\begin{question}
 What is the area of the $k^{th}$ rectangle shown in the figure above? Select all that apply.
  \begin{selectAll}
    \choice{$A=\Delta x $}
     \choice{$A=f(x)\Delta x $}
       \choice{$A= f( x_k^*)x_k^*$}
    \choice[correct]{$A= f( x_k^*)\Delta x$}
    \choice{$A=f(x)x$}  
     \choice{$A=k\Delta k$} 
       \choice[correct]{$A= f( x_k^*)(x_k-x_{k-1})$}
  \end{selectAll}
\end{question} 
Here are three options for sample points that we consider:

\subsection{Rectangles defined by left-endpoints}

We can set the rectangles up so that the left-endpoint touches the
curve.
\begin{image}
  \begin{tikzpicture}[
      declare function = {f(\x) = -sin(deg(\x)) + 3;}]
    \begin{axis}[  
        domain=-.2:7, xmin =-.2,xmax=7,ymax=5,ymin=-.7,
        width=6in,
        height=3in,xtick={1,2,...,5},
        xticklabels={$x_1^*$,$x_2^*$,$x_3^*$,$x_4^*$,$x_5^*$},
        ytick style={draw=none},
        yticklabels={},
        axis lines=center, xlabel=$x$, ylabel=$y$,
        every axis y label/.style={at=(current axis.above origin),anchor=south},
        every axis x label/.style={at=(current axis.right of origin),anchor=west},
        axis on top,
      ]
      \foreach \rectnumber in {1,2,...,5}
               {
                 \addplot [draw=penColor,fill=fillp] plot coordinates
                          {({1+\rectnumber-1) * 5/5},{f(1+(\rectnumber-1) * 5/5)})
                            ({(1+\rectnumber) * 5/5},{f(1+(\rectnumber-1) * 5/5) })} \closedcycle;
               };
               \addplot [very thick,penColor, smooth] {f(x)};
               \node[fillp!50!black] at (axis cs:1.5,1) {\huge$\mathsf{1}$};
               \node[fillp!50!black] at (axis cs:2.5,1) {\huge$\mathsf{2}$};
               \node[fillp!50!black] at (axis cs:3.5,1) {\huge$\mathsf{3}$};
               \node[fillp!50!black] at (axis cs:4.5,1) {\huge$\mathsf{4}$};
               \node[fillp!50!black] at (axis cs:5.5,1) {\huge$\mathsf{5}$}; 
    \end{axis}
  \end{tikzpicture}
\end{image}
In the graph above, the $k^{th}$ rectangle's left-endpoint is touching the curve.





\subsection{Rectangles defined by right-endpoints}

We can set the rectangles up so that the right-endpoint touches the
curve.
\begin{image}
  \begin{tikzpicture}[
      declare function = {f(\x) = -sin(deg(\x)) + 3;}]
    \begin{axis}[  
        domain=-.2:7, xmin =-.2,xmax=7,ymax=5,ymin=-.7,
        width=6in,
        height=3in,xtick={2,3,...,6},
        xticklabels={$x_1^*$,$x_2^*$,$x_3^*$,$x_4^*$,$x_5^*$},
        ytick style={draw=none},
        yticklabels={},
        axis lines=center, xlabel=$x$, ylabel=$y$,
        every axis y label/.style={at=(current axis.above origin),anchor=south},
        every axis x label/.style={at=(current axis.right of origin),anchor=west},
        axis on top,
      ]
      \foreach \rectnumber in {1,2,...,5}
               {
                 \addplot [draw=penColor,fill=fillp] plot coordinates
                          {({1+\rectnumber-1) * 5/5},{f(1+(\rectnumber) * 5/5)})
                            ({(1+\rectnumber) * 5/5},{f(1+(\rectnumber) * 5/5) })} \closedcycle;
               };
               \addplot [very thick,penColor, smooth] {f(x)};
               \node[fillp!50!black] at (axis cs:1.5,1) {\huge$\mathsf{1}$};
               \node[fillp!50!black] at (axis cs:2.5,1) {\huge$\mathsf{2}$};
               \node[fillp!50!black] at (axis cs:3.5,1) {\huge$\mathsf{3}$};
               \node[fillp!50!black] at (axis cs:4.5,1) {\huge$\mathsf{4}$};
               \node[fillp!50!black] at (axis cs:5.5,1) {\huge$\mathsf{5}$}; 
    \end{axis}
  \end{tikzpicture}
\end{image}
In the graph above, the $k^{th}$ rectangle's right-endpoint is touching the curve.


\subsection{Rectangles defined by midpoints}

We can set the rectangles up so that the midpoint of one of the
horizontal sides touches the curve.
\begin{image}
  \begin{tikzpicture}[
      declare function = {f(\x) = -sin(deg(\x)) + 3;}]
    \begin{axis}[  
        domain=-.2:7, xmin =-.2,xmax=7,ymax=5,ymin=-.7,
        width=6in,
        height=3in,xtick={1.5,2.5,...,5.5},
        xticklabels={$x_1^*$,$x_2^*$,$x_3^*$,$x_4^*$,$x_5^*$},
        ytick style={draw=none},
        yticklabels={},
        axis lines=center, xlabel=$x$, ylabel=$y$,
        every axis y label/.style={at=(current axis.above origin),anchor=south},
        every axis x label/.style={at=(current axis.right of origin),anchor=west},
        axis on top,
      ]
      \foreach \rectnumber in {1,2,...,5}
               {
                 \addplot [draw=penColor,fill=fillp] plot coordinates
                          {({1+\rectnumber-1) * 5/5},{f(1+(\rectnumber-.5) * 5/5)})
                            ({(1+\rectnumber) * 5/5},{f(1+(\rectnumber-.5) * 5/5) })} \closedcycle;
                 \addplot [draw=black,dashed] plot coordinates
                                   {({1+\rectnumber-.5) * 5/5},{0 * 5/5)})
                            ({(1+\rectnumber-.5) * 5/5},{f(1+(\rectnumber-.5) * 5/5) })};
               };
               \addplot [very thick,penColor, smooth] {f(x)};
               \node[fillp!50!black] at (axis cs:1.5,1) {\huge$\mathsf{1}$};
               \node[fillp!50!black] at (axis cs:2.5,1) {\huge$\mathsf{2}$};
               \node[fillp!50!black] at (axis cs:3.5,1) {\huge$\mathsf{3}$};
               \node[fillp!50!black] at (axis cs:4.5,1) {\huge$\mathsf{4}$};
               \node[fillp!50!black] at (axis cs:5.5,1) {\huge$\mathsf{5}$}; 
    \end{axis}
  \end{tikzpicture}
\end{image}
In the graph above, the midpoint of the horizontal side of the $k^{th}$
rectangle is touching the curve.

%% \begin{question}
%%   Which approximation is most accurate?
%%   \begin{multipleChoice}
%%     \choice{Approximations with left-endpoints.}
%%     \choice{Approximations with right-endpoints.}
%%     \choice{Approximations with midpoints.}
%%     \choice[correct]{It depends on the function.}
%%   \end{multipleChoice}
%% \end{question}



\section{Riemann sums and approximating area}

Once we know how to identify our rectangles, we can compute some
approximate areas. If we are approximating area with $n$ rectangles, then 
%\begin{align*}
%  \text{Area} &\approx \sum_{k=1}^n (\text{height of $k$th rectangle})\times(\text{width of $k$th rectangle}) \\
%  &=\sum_{k=1}^n  f(x_k^*)\Delta x \\
%  &=  f(x_1^*)\Delta x +  f(x_2^*)\Delta x +   f(x_3^*)\Delta x + \dots +   %f(x_n^*)\Delta x. 
%\end{align*}
the area of the $k$th rectangle (for $k$ between $1$ and $n$) is given by:
\[ (\text{height of $k$th rectangle}) \times (\text{width of $k$th rectangle})
	= f(x_k^*) \Delta x \] 
The area of the region is approximately:
\[  \text{Area} \approx  
	f(x_1^*)\Delta x +  f(x_2^*)\Delta x +   f(x_3^*)\Delta x + \dots +   f(x_n^*)\Delta x.  \]
\begin{definition}
  A sum of the form:
%  \[
%  \sum_{k=1}^n  f(x_k^*)\Delta x  = f(x_1^*)\Delta x +  f(x_2^*)\Delta x + \dots %+   f(x_n^*)\Delta x
%  \]
  \[ f(x_1^*)\Delta x +  f(x_2^*)\Delta x + \dots +   f(x_n^*)\Delta x \]
  is called a \dfn{Riemann sum}, pronounced ``ree-mahn'' sum. 
\end{definition}

A Riemann sum computes an approximation of the area between a curve
and the $x$-axis on the interval $[a,b]$. It can be defined several
different ways. We will focus on it being defined via left-endpoints,
right-endpoints, and midpoints. In addition we will consider Upper and Lower Riemann sums where we want each rectangle to be the tallest and shortest respectively. Here we see the explicit connection
between a Riemann sum defined by left-endpoints and the area between a curve
and the $x$-axis on the interval $[a,b]$:
\begin{image}
  \begin{tikzpicture}[
      declare function = {f(\x) = -sin(deg(\x)) + 3;}]
    \begin{axis}[  
        domain=-.2:7, xmin =-.2,xmax=7,ymax=5,ymin=-.7,
        width=6in,
        height=3in,
        xtick style={draw=none},
        xticklabels={},
        ytick style={draw=none},
        yticklabels={},
        axis lines=center, xlabel=$x$, ylabel=$y$,
        every axis y label/.style={at=(current axis.above origin),anchor=south},
        every axis x label/.style={at=(current axis.right of origin),anchor=west},
        axis on top,
      ]
      \foreach \rectnumber in {1,2,...,5}
               {
                 \addplot [draw=penColor,fill=fillp] plot coordinates
                          {({1+\rectnumber-1) * 5/5},{f(1+(\rectnumber-1) * 5/5)})
                            ({(1+\rectnumber) * 5/5},{f(1+(\rectnumber-1) * 5/5) })} \closedcycle;
                 \addplot[decoration={brace,mirror,raise=.2cm},decorate,thin] plot coordinates
                          {(\rectnumber+.1,0)
                            (1+\rectnumber-.1,0)};
               };
               \addplot [very thick,penColor, smooth] {f(x)};
               \addplot[decoration={brace,raise=.2cm},decorate,thin] plot coordinates
                       {(.95,.1) (.95,{f(1)-.1})};
               \addplot[decoration={brace,mirror,raise=.2cm},decorate,thin] plot coordinates
                       {(6.05,.1) (6.05,{f(5)-.1})};
               \node at (axis cs:1.5,-.5) {$\Delta x$};
               \node at (axis cs:2.5,-.5) {$\Delta x$};
               \node at (axis cs:3.5,-.5) {$\Delta x$};
               \node at (axis cs:4.5,-.5) {$\Delta x$};
               \node at (axis cs:5.5,-.5) {$\Delta x$};

               \node at (axis cs:1.5,1) {\large$f(x_1^*)\Delta x$};
               \node at (axis cs:2.5,1) {\large$f(x_2^*)\Delta x$};
               \node at (axis cs:3.5,1) {\large$f(x_3^*)\Delta x$};
               \node at (axis cs:4.5,1) {\large$f(x_4^*)\Delta x$};
               \node at (axis cs:5.5,1) {\large$f(x_5^*)\Delta x$};

               \node[anchor=east] at (axis cs:.8,{f(1)/2}) {$f(x_1^*)$};
               \node[anchor=west] at (axis cs:6.2,{f(5)/2}) {$f(x_5^*)$};
               
               \node[anchor=south] at (axis cs:1,{f(1)}) {$f(x_1^*)$};
               \node[anchor=south] at (axis cs:2,{f(2)}) {$f(x_2^*)$};
               \node[anchor=south] at (axis cs:3,{f(3)}) {$f(x_3^*)$};
               \node[anchor=south] at (axis cs:4,{f(4)}) {$f(x_4^*)$};
               \node[anchor=south] at (axis cs:5,{f(5)}) {$f(x_5^*)$};
    \end{axis}
  \end{tikzpicture}
\end{image}
and here is the associated Riemann sum
\[
%\sum_{k=1}^5  f(x_k^*)\Delta x  = 
f(x_1^*)\Delta x +  f(x_2^*)\Delta x +   f(x_3^*)\Delta x + f(x_4^*)\Delta x + f(x_5^*)\Delta x.
\]



\subsection{Left Riemann sums}

\begin{example}
  Consider $f(x) = \frac{1}{8}x^3-x+2$. Approximate the area between $f$ and
  the $x$-axis on the interval $[-1,3]$ using a left-endpoint Riemann
  sum with $n=5$ rectangles.
  
  \begin{explanation}
    First note that the width of each rectangle is
    \[
    \Delta x = \frac{3-(-1)}{5} = \answer[given]{4/5}.
    \]
\begin{comment}
    The grid points define the edges of the rectangle and are seen below:
\begin{image}
  \begin{tikzpicture}[
      declare function = {f(\x) = pow(\x/2,3) -\x+2;}
      ]
	\begin{axis}[
            domain=-1.2:3.2, xmin =-1.2,xmax=3.2,ymax=4,ymin=-.2,
            width=6in,
            height=3in,
            axis lines=center, xlabel=$x$, ylabel=$y$,
            every axis y label/.style={at=(current axis.above origin),anchor=south},
            every axis x label/.style={at=(current axis.right of origin),anchor=west},
            xtick={-1,-.2,...,3.8},
            xticklabels={$x_0=-1$,$x_1=-0.2$,$x_2=0.6$,$x_3=1.4$,$x_4=2.2$,$x_5=3$},
            ytick style={draw=none},
            yticklabels={},
            axis on top,
          ]
          \foreach \rectnumber in {1,2,...,5}
                   {
                     \addplot [draw=penColor,fill=fillp] plot coordinates
                              {({-1+(\rectnumber-1) * 4/5},{f(-1+(\rectnumber-1) * 4/5)})
                                ({-1+(\rectnumber) * 4/5},{f(-1+(\rectnumber-1) * 4/5) })} \closedcycle;
               };
                   \addplot [very thick,penColor, smooth] {f(x)};
        \end{axis}
\end{tikzpicture}
\end{image}
\end{comment}
Notice the sample points identify which endpoints we use to determine the height of the rectangle. Notice that each sample point will be the leftmost point in each subinterval $[x_{k-1},x_k]$.
\begin{image}
  \begin{tikzpicture}[
      declare function = {f(\x) = pow(\x/2,3) -\x+2;}
      ]
	\begin{axis}[
            domain=-1.2:3.2, xmin =-1.2,xmax=3.2,ymax=4,ymin=-.2,
            width=6in,
            height=3in,
            axis lines=center, xlabel=$x$, ylabel=$y$,
            every axis y label/.style={at=(current axis.above origin),anchor=south},
            every axis x label/.style={at=(current axis.right of origin),anchor=west},
            xtick={-1,-.2,...,3},
            xticklabels={$x_1^*=-1$,$x_2^*=-0.2$,$x_3^*=0.6$,$x_4^*=1.4$,$x_5^*=2.2$},
            ytick style={draw=none},
            yticklabels={},
            axis on top,
          ]
          \foreach \rectnumber in {1,2,...,5}
                   {
                     \addplot [draw=penColor,fill=fillp] plot coordinates
                              {({-1+(\rectnumber-1) * 4/5},{f(-1+(\rectnumber-1) * 4/5)})
                                ({-1+(\rectnumber) * 4/5},{f(-1+(\rectnumber-1) * 4/5) })} \closedcycle;
               };
                   \addplot [very thick,penColor, smooth] {f(x)};
                   \node[fillp!50!black] at (axis cs:-.6,.4) {\huge$\mathsf{1}$};
               \node[fillp!50!black] at (axis cs:.2,.4) {\huge$\mathsf{2}$};
               \node[fillp!50!black] at (axis cs:1,.4) {\huge$\mathsf{3}$};
               \node[fillp!50!black] at (axis cs:1.8,.4) {\huge$\mathsf{4}$};
               \node[fillp!50!black] at (axis cs:2.6,.4) {\huge$\mathsf{5}$}; 
        \end{axis}
\end{tikzpicture}
\end{image}
It is helpful to collect all of this data into a table:
\[
\begin{array}{c|c|c}
\text{index}&\text{sample point}&\text{rectangle height}\\
  k &  x^*_k=x_{k-1} & f(x^*_k) \\ \hline
  1 &  \answer[given]{-1}        &   \answer[given]{2.875}     \\
  2 &  \answer[given]{-0.2}      & \answer[given]{2.199}   \\
  3 &  \answer[given]{0.6}       & \answer[given]{1.427}    \\
  4 & \answer[given]{1.4}       & \answer[given]{0.943}   \\
  5 &  \answer[given]{2.2}      & \answer[given]{1.131}        \\
\end{array}
\]
Now we may write a left Riemann sum and approximate the area
\[
f(x_1^*)\Delta x +  f(x_2^*)\Delta x +   f(x_3^*)\Delta x +f(x_4^*)\Delta x+   f(x_5^*)\Delta x,
\]
which evaluates to
\begin{align*}
  = 2.875 \cdot (4/5) &+ 2.199  \cdot(4/5) + 1.427  \cdot(4/5)\\
  &+ 0.943  \cdot(4/5) + 1.131  \cdot(4/5)
\end{align*}
and we find
\[
  = \answer[given]{6.86}.
\]
  \end{explanation}
\end{example}


\subsection{Right Riemann sums}


\begin{example}
  Consider $f(x) = \frac{1}{8}x^3-x+2$. Approximate the area between $f$ and
  the $x$-axis on the interval $[-1,3]$ using a right-endpoint Riemann
  sum with $n=5$ rectangles.
  
  \begin{explanation}
    First note that the width of each rectangle is
    \[
    \Delta x = \frac{3-(-1)}{5} = \answer[given]{4/5}.
    \]
  
Notice the sample points identify which endpoints we use. Notice in this case we use the rightmost point in each subinterval $[x_{k-1},x_k]$
\begin{image}
  \begin{tikzpicture}[
      declare function = {f(\x) = pow(\x/2,3) -\x+2;}
      ]
	\begin{axis}[
            domain=-1.2:3.2, xmin =-1.2,xmax=3.2,ymax=4,ymin=-.2,
            width=6in,
            height=3in,
            axis lines=center, xlabel=$x$, ylabel=$y$,
            every axis y label/.style={at=(current axis.above origin),anchor=south},
            every axis x label/.style={at=(current axis.right of origin),anchor=west},
            xtick={-.2,.6,1.4,2.2,3},
            xticklabels={$x_1^*=-0.2$,$x_2^*=0.6$,$x_3^*=1.4$,$x_4^*=2.2$,$x_5^*=3$},
            ytick style={draw=none},
            yticklabels={},
            axis on top,
          ]
          \foreach \rectnumber in {1,2,...,5}
                   {
                     \addplot [draw=penColor,fill=fillp] plot coordinates
                              {({-1+(\rectnumber-1) * 4/5},{f(-1+(\rectnumber) * 4/5)})
                                ({-1+(\rectnumber) * 4/5},{f(-1+(\rectnumber) * 4/5) })} \closedcycle;
               };
                   \addplot [very thick,penColor, smooth] {f(x)};
                   \node[fillp!50!black] at (axis cs:-.6,.4) {\huge$\mathsf{1}$};
               \node[fillp!50!black] at (axis cs:.2,.4) {\huge$\mathsf{2}$};
               \node[fillp!50!black] at (axis cs:1,.4) {\huge$\mathsf{3}$};
               \node[fillp!50!black] at (axis cs:1.8,.4) {\huge$\mathsf{4}$};
               \node[fillp!50!black] at (axis cs:2.6,.4) {\huge$\mathsf{5}$}; 
        \end{axis}
\end{tikzpicture}
\end{image}
It is helpful to collect all of this data into a table:
\[
\begin{array}{c|c|c}
\text{index}&\text{sample point}&\text{rectangle height}\\
  k &  x^*_k=x_k & f(x^*_k) \\ \hline
  1 &  \answer[given]{-0.2}      & \answer[given]{2.199}   \\
  2 &  \answer[given]{0.6}       & \answer[given]{1.427}   \\
  3 &  \answer[given]{1.4}       & \answer[given]{0.943}   \\
  4 &  \answer[given]{2.2}       & \answer[given]{1.131}   \\
  5 & \answer[given]{3}         & \answer[given]{2.375}   \\
\end{array}
\]
Now we may write a right Riemann sum and approximate the area
\[
f(x_1^*)\Delta x +  f(x_2^*)\Delta x +   f(x_3^*)\Delta x +f(x_4^*)\Delta x+   f(x_5^*)\Delta x,
\]
which evaluates to
\begin{align*}
  = 2.199 \cdot (4/5) &+ 1.427  \cdot(4/5) + 0.943  \cdot(4/5)\\
  &+ 1.131  \cdot(4/5) + 2.375  \cdot(4/5)
\end{align*}
and we find
\[
  = \answer[given]{6.46}.
\]
  \end{explanation}
\end{example}




\subsection{Midpoint Riemann sums}

\begin{example}
  Consider $f(x) = x^3/8-x+2$. Approximate the area between $f$ and
  the $x$-axis on the interval $[-1,3]$ using a midpoint Riemann
  sum with $n=5$ rectangles.
  
  \begin{explanation}
    First note that the width of each rectangle is
    \[
    \Delta x = \frac{3-(-1)}{5} = \answer[given]{4/5}.
    \]
   
Again, the sample points identify which point we use. In this case we use the midpoint of each subinterval $[x_{k-1},x_k]$.
\begin{image}
  \begin{tikzpicture}[
      declare function = {f(\x) = pow(\x/2,3) -\x+2;}
      ]
	\begin{axis}[
            domain=-1.2:3.2, xmin =-1.2,xmax=3.2,ymax=4,ymin=-.2,
            width=6in,
            height=3in,
            axis lines=center, xlabel=$x$, ylabel=$y$,
            every axis y label/.style={at=(current axis.above origin),anchor=south},
            every axis x label/.style={at=(current axis.right of origin),anchor=west},
            xtick={-.6,.2,1,1.8,2.6},
            xticklabels={$x_1^*=-0.6$,$x_2^*=0.2$,$x_3^*=1$,$x_4^*=1.8$,$x_5^*=2.6$},
            ytick style={draw=none},
            yticklabels={},
            axis on top,
          ]
          \foreach \rectnumber in {1,2,...,5}
                   {
                     \addplot [draw=penColor,fill=fillp] plot coordinates
                              {({-1+(\rectnumber-1) * 4/5},{f(-1+(\rectnumber-.5) * 4/5)})
                                ({-1+(\rectnumber) * 4/5},{f(-1+(\rectnumber-.5) * 4/5) })} \closedcycle;
                     \addplot [draw=black,dashed] plot coordinates
                                   {({-1+(\rectnumber-.5) * 4/5},{0 * 4/5)})
                            ({(-1+(\rectnumber-.5) * 4/5},{f(-1+(\rectnumber-.5) * 4/5) })};
               };
                   \addplot [very thick,penColor, smooth] {f(x)};
                   \node[fillp!50!black] at (axis cs:-.6,.4) {\huge$\mathsf{1}$};
               \node[fillp!50!black] at (axis cs:.2,.4) {\huge$\mathsf{2}$};
               \node[fillp!50!black] at (axis cs:1,.4) {\huge$\mathsf{3}$};
               \node[fillp!50!black] at (axis cs:1.8,.4) {\huge$\mathsf{4}$};
               \node[fillp!50!black] at (axis cs:2.6,.4) {\huge$\mathsf{5}$}; 
        \end{axis}
\end{tikzpicture}
\end{image}
It is helpful to collect all of this data into a table:
\[
\begin{array}{c|c|c}
\text{index}&\text{sample point}&\text{rectangle height}\\
  k &   x^*_k=(x_{k-1}+x_k)/2 & f(x^*_k) \\ \hline
  1 &  \answer[given]{-0.6}    & \answer[given]{2.573}   \\
  2 &  \answer[given]{0.2}     & \answer[given]{1.801}   \\
  3 &  \answer[given]{1}       & \answer[given]{1.125}   \\
  4 & \answer[given]{1.8}     & \answer[given]{0.929}   \\
  5 & \answer[given]{2.6}     & \answer[given]{1.597}   \\
\end{array}
\]
Now we may write a midpoint Riemann sum and approximate the area
\[
f(x_1^*)\Delta x + f(x_2^*)\Delta x + f(x_3^*)\Delta x+ f(x_4^*)\Delta x + f(x_5^*)\Delta x,
\]
which evaluates to
\begin{align*}
  = 2.573 \cdot (4/5) &+ 1.801\cdot(4/5) + 1.125 \cdot(4/5) \\
  &+ 0.929 \cdot(4/5) + 1.597\cdot(4/5)
\end{align*}
and we find
\[
= \answer[given]{6.42}.
\]
  \end{explanation}
\end{example}



\subsection{Summary}

Riemann sums approximate the area between curves and the $x$-axis via
rectangles.  When computing this area via rectangles, there are
several things to know:
\begin{itemize}
\item What interval are we on? In our discussion above we call this
    $[a,b]$.
  \item How many rectangles will be used? In our discussion above we
    called this $n$.
  \item What is the width of each individual rectangle? In our discussion above we
    called this $\Delta x$.
  \item What points will determine the height of the rectangle? In our
    discussion above we called these sample points, $x_k^*$, and they
    can be left-endpoints, right-endpoints, or midpoints.
  \item What is the actual height of the rectangle? This will always
    be $f(x_k^*)$.
\end{itemize}



\end{document}