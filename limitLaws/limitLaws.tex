\documentclass{ximera}


\outcome{Calculate limits using the limit laws.}

\input{../preamble.tex}
\title{The Limit Laws and Squeeze Theorem}
\begin{document}
\begin{abstract}
We give basic laws for working with limits. 
\end{abstract}
\maketitle
\section{Limit Laws}
In the previous section were able to compute the limits

\[
\lim_{x\to 3} x^3=27,\quad\lim_{x\to 3} \sqrt{2}=\sqrt{2},\quad\lim_{x\to 3} 2^x=8,
\]

using continuity of the functions $x^3$, $\sqrt{2}$, and $2^x$, at $x=3$.  Does this imply that we can compute the limits
\begin{align*}
  &\lim_{x\to 3} (x^3-2^x),\\
  &\lim_{x\to 3} \frac{\sqrt{2}}{2^x},\\
  &\lim_{x\to 3} (\sqrt{2}\cdot x^3\cdot 2^x), \text{ or}\\
  &\lim_{x\to 3} 2^{x^3}?
\end{align*}

Well, we cannot use continuity here, because we don't know if the
functions $x^3-2^x$, $\frac{\sqrt{2}}{2^x}$, $\sqrt{2}\cdot
x^3\cdot 2^{x}$, and $2^{x^3}$ are continuous at $x=3$, and

 we have no other tools available, since the graphs and tables are not reliable. Obviously, we need more tools to help us with computation of limits.

 

In this section, we present a handful of rules, called the \textit{Limit Laws},
that allow us to find limits of various combinations of functions.

\begin{theorem}[Limit Laws]\index{limit laws}\label{theorem:limit-laws}
Suppose that $\lim_{x\to a}f(x)=L$, $\lim_{x\to a}g(x)=M$, $n$ and $m$ are positive integers, and $k$ a real number.
\begin{description}
\item[\textbf{Constant Multiple Law}] $\lim_{x\to a} kf(x) = k\lim_{x\to a}f(x)=kL$.
\item[Sum/Difference Law] $\lim_{x\to a} (f(x) \pm g(x)) =
  \lim_{x\to a}f(x) \pm \lim_{x\to a}g(x)=L \pm M$.
\item[Product Law]  $\lim_{x\to a} (f(x)g(x)) = \lim_{x\to
  a}f(x)\cdot\lim_{x\to a}g(x)=LM$.
\item[Quotient Law]  $\lim_{x\to a} \frac{f(x)}{g(x)} =
  \frac{\lim_{x\to a}f(x)}{\lim_{x\to a}g(x)}=\frac{L}{M}$, if
  $M\ne0$.
\item[Power Law] $\lim_{x\to a} (f(x))^n=(\lim_{x\to a} f(x))^n=L^n$.
\item[Root Law] $\lim_{x\to a} \sqrt[m]{f(x)}=\sqrt[m]{\lim_{x\to a} f(x)}=\sqrt[m]{L}$, if $m$ is even, then $L$ must be a non-negative real number.
\end{description}
\label{thm:limit laws}
\end{theorem}

In plain language, ``limit of a sum equals the sum of the limits,'' ``limit of a product equals the product of the limits,'' etc.

Let's examine how the Limit Laws can be used in computation of limits.
%begin{question}
  %True or false: If $f$ and $g$ are continuous functions on an
  %interval $I$, then $f\pm g$ is continuous on $I$.
  %\begin{multipleChoice}
   % \choice[correct]{True}
  %  \choice{False}
 % \end{multipleChoice}
 % \begin{feedback}
    %This follows from the Sum/Difference Law.
 % \end{feedback}
%\end{question}

%\begin{question}
 % True or false: If $f$ and $g$ are continuous functions on an
 % interval $I$, then $f/g$ is continuous on $I$.
 % \begin{multipleChoice}
    %\choice{True}
   % \choice[correct]{False}
 % \end{multipleChoice}
 % \begin{feedback}
   % In this case, $f/g$ will not be continuous for $x$ where $g(x) =
    %0$.
 % \end{feedback}
%\end{question}


\begin{example}
  Compute the following limits using Limit Laws:
  \begin{enumerate}
  \item\label{lle1a} $\lim_{x\to \pi} (x^3-2^x)$
  \item\label{lle1b} $\lim_{x\to \pi} \frac{\sqrt{2}}{2^x}$
  \item\label{lle1c} $\lim_{x\to \pi} (\sqrt{2}\cdot x^3\cdot 2^{x})$
  \item\label{lle1d} $\lim_{x\to \pi} 2^{x^3}$
  \end{enumerate}
  \begin{explanation}
    For $\lim_{x\to 3} (x^3-2^x)$, write:
    \begin{align*}
      \lim_{x\to 3} (x^3-2^x)&=\lim_{x\to 3} x^3-\lim_{x\to 3}2^x && \text{by Difference Law}\\
      &=27-8\\
      &=19
    \end{align*}
    For $\lim_{x\to 3} \frac{\sqrt{2}}{2^x}$, write:
    \begin{align*}
      \lim_{x\to 3} \frac{\sqrt{2}}{2^x}&=\frac{\lim_{x\to 3}\sqrt{2}}{\lim_{x\to 3}2^x} && \text{by Quotient Law}\\
      &=\frac{\sqrt{2}}{8}
    \end{align*}
    For $\lim_{x\to 3} (\sqrt{2}\cdot x^3\cdot 2^x)$, using the Product Law, we can write:
    \[
    \lim_{x\to 3} (\sqrt{2}\cdot x^3\cdot 2^x)=\lim_{x\to 3} \sqrt{2}\cdot \lim_{x\to 3}x^3\cdot \lim_{x\to 3}2^x
    \]
    \begin{align*}
      &=\sqrt{2}\cdot 27\cdot 8\\
      &=216\sqrt{2}
    \end{align*}
    For $\lim_{x\to 3} 2^{x^3}$, write:
    \[
    \lim_{x\to 3} 2^{x^3}= \cdots 
    \]
    The function $2^{x^3}$ is a combination of the functions
    $2^x$ and $x^3$, but it is neither a sum/difference, nor a
    product, nor a quotient of these two functions, so we cannot apply
    any of the Limit Laws. This function is a composition of the two
    functions, $2^x$ and $x^3$.
  \end{explanation}
\end{example}
\begin{example}
\begin{enumerate}
\item  Compute the following limit using limit laws:
\[
  \lim_{x\to 1}(5x^2+3x-2)
\]
\begin{explanation}
  Well, let's start by splitting it up:
  \[
  \lim_{x\to 1} (5x^2+3x-2) = \lim_{x\to 1} 5x^2 + \lim_{x\to 1} \answer[given]{3x} - \lim_{x\to 1}2
  \]
  by the Sum/Difference Law. So now
  \[
  = 5\lim_{x\to 1} x^2 + 3\lim_{x\to 1} x - \lim_{x\to 1}\answer[given]{2}
  \]
  by the Constant Multiple Law. Finally by continuity of $x^b$ and $k$,
  \[
  = 5(1)^2 + 3(1) - 2 =\answer[given]{6}.
  \]
  
  %\begin{onlineOnly}
  %  We can check our answer by looking at the graph of $y=f(x)$:
  %  \[
  %  \graph{5x^2+3x-2}
  %  \]
  %\end{onlineOnly}
\end{explanation}  
\item Is the polynomial function  $f(x)=5x^2+3x-2$ continuous at $x=1$?
\begin{explanation}
We have to check whether
\[
 \lim_{x\to 1} f(x)=f(1). 
\]
Since we already know that $ \lim_{x\to 1} (5x^2+3x-2)=6$,  we only have  to compute $f(1)$.
\begin{align*}
f(1)&=5(1)^2 + 3(1) - 2\\
&=6
\end{align*}
Therefore, the polynomial function $f$ is continuous at $x=1$.

But what about continuity at any other value $x$? Is the function $f$ continuous on its entire domain?

And what about any other polynomial function?

Are polynomials continuous on their domains?
\end{explanation}
\end{enumerate}
\end{example}

We can generalize the example above to get the following theorems.

\begin{theorem}\label{polycont}[Continuity of Polynomial Functions]\index{continuity of polynomial functions}
 \textbf{All polynomial functions}, meaning functions of the form
  \[
  f(x) = a_nx^n + a_{n-1}x^{n-1} + \dots + a_1 x + a_0
  \]
  where $n$ is a positive integer  and each coefficient $a_i$, $i=0, 1,...,n$, is a real number, are
  \textbf{continuous for all real numbers}.

The proof for this theorem is a generalization of the Example \ref{polycont}
  %%%%%%%%%%%%%%%%%%%%%%%
 %% MAKE THIS OPTIONAL DISPLAY/UNDISPLAY IF POSSIBLE?????
 %%%%%%%%%%%%%%%%%%%%%%%%%
 % \begin{explanation}
 % In order to show that any polynomial, $f$, is continuous at any real number, $a$, we have to show that
 % \[
  %\lim_{x\to a} f(x)=f(a).
  %\]
  %Write with me:
  %\[
  %\lim_{x\to a} f(x) = \lim_{x\to a} (a_nx^n + a_{n-1}x^{n-1} + \dots + a_1 x + a_0 )
  %\]
  %Now by the Sum Law,
  %\[
  %= \lim_{x\to a} a_nx^n + \lim_{x\to a} a_{n-1}x^{n-1} + \dots +  \lim_{x\to a}a_1 x + \lim_{x\to a} a_0
  %\]
  %and by the Constant Multiple Law,
  %\begin{align*}
  %  =  a_n\cdot \lim_{x\to a}x^n + a_{n-1}\cdot \lim_{x\to a}x^{n-1} &+ \dots\\
    % &+  a_1 \cdot \lim_{x\to a}x + \lim_{x\to a} a_0
  %\end{align*}
  %and by Continuity
  %\[
  %= a_n\cdot a^n +  a_{n-1}\cdot a^{n-1} + \dots + a_1 \cdot a + a_0
  %\]
 %Which is equal to $f(a)$.

 % Since we have shown that $\lim_{x\to a} f(x) = f(a)$, we have
 % shown that $f$ is continuous at $x=a$.
%\end{explanation}
\end{theorem}

\begin{theorem}[Continuity of Rational Functions]\index{continuity of rational functions}
   A \textbf{rational function} h, meaning a function of the form 
  \[
  h(x)=\frac{f(x)}{g(x)}
  \]
  where $f $and $g$ are polynomials, is \textbf{continuous} for all real numbers except where $g(x)=0$.  That is,
  rational functions are continuous wherever they are defined.
\begin{explanation}
      Let $a$ be a real number such that $g(a)\neq 0$.  Then, since
      $g(x)$ is continuous at $a$, $\lim_{x\to a} g(x) \neq 0$.
      Therefore, 
      \[
      \lim_{x \to a} h(x) = \lim_{x\to a} \frac{f(x)}{g(x)}
      \]
      and now by the Quotient Law, 
      \[
      \frac{\lim_{x\to a} f(x)}{ \lim_{x\to a} g(x)}
      \]
      and by the continuity of polynomials we know that each limit is equal to the function value when $x=a$
      \[
      \frac{f(a)}{g(a)}=h(a).
      \]
      Since we have shown that $\lim_{x\to a} h(x) = h(a)$, we have
      shown that $h$ is continuous at $x=a$.
\end{explanation}
\end{theorem}

\begin{question}
  Where is $f(x) = \frac{x^2-3x+2}{x-2}$ continuous?
  \begin{prompt}
  \begin{multipleChoice}
    \choice{for all real numbers}
    \choice{at $x=2$}
    \choice[correct]{for all real numbers, except $x=2$}
    \choice{impossible to say}
  \end{multipleChoice}
  \end{prompt}
\end{question}
\begin{question}
  True or false: If $f$ and $g$ are continuous functions on an
  interval $I$, then $f\pm g$ is continuous on $I$.
  \begin{prompt}
  \begin{multipleChoice}
    \choice[correct]{True}
    \choice{False}
  \end{multipleChoice}
  \begin{feedback}
     Let's assume that $I$ is an open interval and $a$ is a number in $I$. Remember, since $f$ and $g$ are both continuous on $I$, they are both continuous at $a$. 
     
     This means that $ \lim_{x\to a}f(x)=f(a)$ and $ \lim_{x\to a}g(x)=g(a)$.
     
      Now, define a new function, $h$, where $h(x)= f(x)+g(x)$, for all $x$ in $I$. We have to show that $h$ is continuous at $a$, or that
    
    \[
    \lim_{x\to a} h(x) = h(a).
    \]
    Let's start with
     
    \[
    \lim_{x\to a} h(x) =  \lim_{x\to a} (f(x)+g(x)) = \lim_{x\to a} f(x)+\lim_{x\to a}g(x)\hspace{0.04in}by \hspace{0.04in}Sum\hspace{0.04in} Law,
    \]
    and, therefore,
    
     \[
    \lim_{x\to a} h(x) =  f(a)+g(a)=h(a) \checkmark
    \]
We have proved that $h$ is continuous at  any number $a$ in $I$. Therefore, $h$ is continuous on $I$.
Similarly, we can prove that $f+g$ is continuous on any interval $I$, by showing it is left-or right-continuous at the endpoints.
We can adjust the proof for the function $f-g$.
  \end{feedback}
  \end{prompt}
\end{question}

\begin{question}
  True or false: If $f$ and $g$ are continuous functions on an
  interval $I$, then $f/g$ is continuous on $I$.
  \begin{prompt}
  \begin{multipleChoice}
    \choice{True}
    \choice[correct]{False}
  \end{multipleChoice}
  \begin{feedback}
    In this case, $f/g$ will not be continuous for $x$ where $g(x) =
    0$.
  \end{feedback}
  \end{prompt}
\end{question}


We still don't know how to compute  a limit of a composition of two functions.
Our next theorem provides basic rules for how limits interact with composition
of functions.

\begin{theorem}[Composition Limit Law]\index{composition limit law}
  If $f(x)$ is continuous at $b = \lim_{x\to a} g(x)$, then
  \[
  \lim_{x\to a} f(g(x)) = f(\lim_{x\to a} g(x)).
  \]
\end{theorem}

Because the limit of a continuous function is the same as the function
value, we can now pass limits inside continuous functions.

\begin{corollary}[Continuity of Composite Functions]

If $g$ is continuous at $x=a$, and if $f $ is continuous at $g(a)$,  then $f(g(x))$ is continuous at $x=a$.



\end{corollary}
Using the Composition Limit Law, we can compute the last example from the beginning of this section.
\begin{example}
  Compute the following limit using limit laws:
  \[
  \lim_{x \to 3}2^{x^3}
  \]
  \begin{explanation}
  We will use the Composition Limit Law. 
Let
 \[ 
 f(g(x))=2^{x^3},
 \]
 where $f(x)=2^x$, and $g(x)=x^3$.
  Now, continuity of $x^3$ implies that $\lim_{x \to 3}g(x)=3^3=27$.
    Continuity of $2^x$ implies that $f$ is continuous at $27=\lim_{x \to 3}g(x)$.
     Now, the Composition Limit Law implies that
    \[
    \lim_{x \to 3}2^{x^3}= 2^{\left({\lim_{x \to 3} x^3}\right)}=2^{27}=134,217,728.
    \]

  \end{explanation}
\end{example}

Many of the Limit Laws and theorems about continuity in this section
might seem like they should be obvious.  You may be wondering why we
spent an entire section on these theorems.  The answer is that these
theorems will tell you exactly when it is easy to find the value of a
limit, and exactly what to do in those cases.

The most important thing to learn from this section is whether the
limit laws can be applied for a certain problem, and when we need to
do something more interesting.  We will begin discussing those more
interesting cases in the next section.  
\section{A list of questions}

Let's try this out.

\begin{question}
  Can this limit be directly computed by limit laws?
  \[
  \lim_{x\to 2}\frac{x^2+3x+2}{x+2} 
  \]
  \begin{prompt}
  \begin{multipleChoice}
    \choice[correct]{yes}
    \choice{no}
  \end{multipleChoice}
  \begin{question}
    Compute:
    \[
    \lim_{x\to 2}\frac{x^2+3x+2}{x+2}\begin{prompt} =\answer{3}\end{prompt}
    \]
    \begin{feedback}
      Since $f(x)=\frac{x^2+3x+2}{x+2}$ is a rational function, and
      the denominator does not equal $0$, we see that $f(x)$ is
      continuous at $x=2$.  Thus, to find this limit, it suffices to
      plug $2$ into $f(x)$.
    \end{feedback}
  \end{question}
  \end{prompt}
\end{question}


\begin{question}
  Can this limit be directly computed by limit laws?
  \[
  \lim_{x\to 2}\frac{x^2-3x+2}{x-2}
  \]
  \begin{prompt}
  \begin{multipleChoice}
    \choice{yes}
    \choice[correct]{no}
  \end{multipleChoice}
  \begin{feedback}
    $f(x) = \frac{x^2-3x+2}{x-2}$ is a rational function, but the
    denominator $x-2$ equals $0$ when $x=2$. None of our current
    theorems address the situation when the denominator of a fraction
    approaches $0$.
  \end{feedback}
  \end{prompt}
\end{question}


\begin{question}
  Can this limit be directly computed by limit laws?
  \[
  \lim_{x\to0}\frac{2^x-1}{3^{x-1}}
  \]
  \begin{prompt}
  \begin{multipleChoice}
    \choice[correct]{yes}
    \choice{no}
  \end{multipleChoice}
  \begin{question}
    Compute:
    \[
    \lim_{x\to0}\frac{2^x-1}{3^{x-1}}\begin{prompt} =\answer{0}\end{prompt}
    \]
    \begin{feedback}
      If we are trying to use limit laws to compute this limit, we
      would have to use the Quotient Law to say that
      \[
      \lim_{x\to 0}\frac{2^x-1}{3^{x-1}} = \frac{\lim_{x\to
          0}(2^x-1)}{\lim_{x\to 0}(3^{x-1})}.
      \]
      We are only allowed to use this law if both limits exist and the
      denominator does not equal $0$.  Let's check each limit
      separately, starting with the denominator
      \begin{align*}
        \lim_{x\to 0}(3^{x-1}) &=3^{\lim_{x\to0}(x-1)}\\
        &=3^{-1}\\
        &=\frac{1}{3}
      \end{align*}

      On the other hand the limit in the numerator is
      \begin{align*}
        \lim_{x\to 0}(2^x-1) &=\lim_{x\to0}(2^x)-\lim_{x\to0}(1)\\
        &=1-1\\
        &=0
      \end{align*}
      The limits in both the numerator and denominator exist and the
      limit in the denominator does not equal $0$, so we can use the
      Quotient Law.  We find:
      \[
        \frac{\lim_{x\to 0}(2^x-1)}{\lim_{x\to 0}(3^{x-1})}
        =\frac{0}{\frac{1}{3}}=0.
        \]
    \end{feedback}
  \end{question}
  \end{prompt}
\end{question}


\begin{question}
  Can this limit be directly computed by limit laws?
  \[
  \lim_{x\to 0}(1+x)^{1/x}
  \]
  \begin{prompt}
  \begin{multipleChoice}
    \choice{yes}
    \choice[correct]{no}
  \end{multipleChoice}
  \begin{feedback}
  We do not have any limit laws for functions of the form $f(x)^{g(x)}$, so we cannot compute this limit.
  \end{feedback}
  \end{prompt}
\end{question}
\section{Squeeze Theorem}

In mathematics, sometimes we can study complex functions by relating
them for simpler functions. The \textit{Squeeze Theorem} tells us one
situation where this is possible.

\begin{theorem}[Squeeze Theorem]\index{Squeeze Theorem}
  Suppose that
  \[
  g(x) \le f(x) \le h(x)
  \]
  for all $x$ close to $a$ but not necessarily equal to $a$. If
  \[
  \lim_{x\to a} g(x) = L = \lim_{x\to a} h(x),
  \]
  then $\lim_{x\to a} f(x) = L$.
\end{theorem}

\begin{question}
  Let's consider some function $f$. We know that for all $x$
  \[
  0 \le f(x) \le x^2.
  \]
  What is $\lim_{x\to 0} f(x)$?
  \begin{prompt}
  \begin{multipleChoice}
    \choice{$f(x)$}
    \choice{$f(0)$}
    \choice[correct]{$0$}
    \choice{impossible to say}
  \end{multipleChoice}
  \end{prompt}
  \begin{explanation}
      We know that $0 \le f(x) \le x^2$ and can see that $\lim_{x\to 0} 0=0$ and $\lim_{x\to 0} x^2=0$. Thus, by the squeeze theorem, $\lim_{x\to 0} f(x)=0$.
  \end{explanation}
\end{question}

\begin{question}
    Now, let's consider a function $g$. In this case, we know for all $x$, $x^2-1\leq g(x)\leq x^3$ and $g(1)=2$. What is $\lim_{x\to 1}g(x)$?

 \begin{prompt}
  \begin{multipleChoice}
    \choice{$0$}
    \choice{$1$}
    \choice{$2$}
    \choice[correct]{impossible to say}
  \end{multipleChoice}
  \end{prompt}
  \begin{explanation}
      Remember, that when evaluating a limit, it's all about what's happening close to our $x$-value being considered ($x=1$ in this case), not what is happening at that $x$-value. Thus the fact that $g(1)=2$ is irrelevant. Since we know that $x^2-1\leq g(x)\leq x^3$. However, $\lim_{x\to 1} x^2-1=0$ and $\lim_{x\to 1} x^3=1$. Thus, we do not have the needed hypotheses to use the squeeze theorem. So we do not have the information to determine $\lim_{x\to 1}g(x)$.
  \end{explanation}
\end{question}

Learning Objectives:

The following are \textbf{facts} or \textbf{definitions} you must memorize from this section:
\begin{itemize}
\item The Limit Laws as listed in Theorem \ref{theorem:limit-laws} 
\item Polynomials are continuous for all real numbers.
\item Rational functions are continuous for all numbers in its domain.
\item A composition of function, $f(g(x))$ is continuous if the inside function, $g(x)$, is continuous at $x=a$ and the outside function, $f(x)$ is continuous at $g(a)$. 
\item The squeeze theorem: Suppose that
  \[
  g(x) \le f(x) \le h(x)
  \]
  for all $x$ close to $a$ but not necessarily equal to $a$. If
  \[
  \lim_{x\to a} g(x) = L = \lim_{x\to a} h(x),
  \]
  then $\lim_{x\to a} f(x) = L$.
\end{itemize}


The following are \textbf{procedures} you must be able to perform from this section:
\begin{itemize}
\item Evaluate limits of combined functions using limit laws. 
\item Determine continuity of combined functions
\item Determine when you can evaluate a limit using limit laws.
\item Use the Squeeze Theorem to evaluate limits when appropriate.

\end{itemize}


\end{document}