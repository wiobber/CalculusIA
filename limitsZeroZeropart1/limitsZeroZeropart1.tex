\documentclass{ximera}

\input{../preamble.tex}


\outcome{Calculate limits of the form zero over zero.}

\title{2.4 - Limits of the form zero over zero, part 1}

\begin{document}
\begin{abstract}
\end{abstract}
\maketitle

So far, we have looked at a few ways to evaluate a limit:

\begin{itemize}
\item If a graph is given, we can use the behavior of the graph to determine the value of the limit.
\item If we know the function is continuous at a value, we can simply plug the value into the function to determine the limit.
\end{itemize}

In this section, we will investigate limits at values which are \textbf{not} in the domain of the given function. Therefore, the function is guaranteed to be discontinous at these values. Specifically, in this section, we will be interested in limits of a fraction whose numerator and denominator are both approaching $0$.

An important observations about limits: Suppose we are looking at $\displaystyle \lim_{x\to a} f(x)$. Recall that a limit only ``sees" the behavior of the function AROUND $a$, not specifically at $a$. As we have observed, even if $a$ is not in the domain of $f(x)$, the limit still may exist. Thus, \textbf{if we can make an algebraic simplification that is valid everywhere except $a$, that algebraic simplification is allowed in the limit.}

One more note: as you already know, division by zero is undefined. It does not mean anything. If $f(a)$ produces $\displaystyle \frac{\text{anything}}{0}$, then the function value itself is \textbf{undefined}. The limit may still exist despite this.

\begin{problem}
  Consider the function
  \[
  f(x) = \frac{x}{x}.
  \]
  \[
  f(0) = \answer{undefined}\qquad\lim_{x\to 0} f(x) = \answer{1}.
  \]
\end{problem}

\begin{problem}
  Consider the function
  \[
  f(x) = \frac{4x}{x}.
  \]
  \[
  f(0) = \answer{undefined}\qquad\lim_{x\to 0} f(x) = \answer{4}.
  \]
\end{problem}

\begin{problem}
  Consider the function
  \[
  f(x) = \frac{x}{-3x}.
  \]
  \[
  f(0) = \answer{undefined}\qquad\lim_{x\to 0} f(x) = \answer{-1/3}.
  \]
\end{problem}

What does this mean? Notice that the form we saw in each of these was the same. When we try plugging in the value,we get the form zero over zero. But the limit value was different in each case. The lesson we can learn from this is that \textbf{if the form of the limit is zero over zero, the limit could be any value.} An algebraic simplification can help us figure out what the value of the limit is.

Let's talk a bit more about the form zero over zero:

\[
\lim_{x\to 2}\frac{x^2-3x+2}{x-2}.
\]
Here 
\[
\lim_{x\to 2}\left(x^2-3x+2\right) = 0\qquad\text{and}\qquad \lim_{x\to
  2}\left(x-2\right) = 0
\]
in light of this, you may think that the limit is one or
zero. \textbf{Not so fast}. This limit is of an \textit{indeterminate
  form}. 

\begin{definition}
  A limit
  \[
  \lim_{x\to a} \frac{f(x)}{g(x)}
  \]
  is said to be of the form \zeroOverZero\ if
  \[
  \lim_{x\to a} f(x) = 0\qquad\text{and}\qquad \lim_{x\to a} g(x) =0.
  \]
\end{definition}

\begin{question}
  Which of the following limits are of the form \zeroOverZero?
  \begin{selectAll}
    \choice{$\lim_{x\to 0}\frac{x^2-3x+2}{x-2}$}
    \choice[correct]{$\lim_{x\to 2}\frac{x^2-3x+2}{x-2}$}
    \choice{$\lim_{x\to 3}\frac{x^2-3x+2}{x-3}$}
  \end{selectAll}
\end{question}

\begin{warning}
  The symbol \zeroOverZero\ is \textbf{not} the number $0$ divided by
  $0$. It is simply short-hand and means that a limit $\lim_{x\to a}
  \frac{f(x)}{g(x)}$ has the property that
  \[
  \lim_{x\to a} f(x) = 0\qquad\text{and}\qquad \lim_{x\to a} g(x) =0.
  \]
\end{warning}


Let's consider an example with the function above:

\begin{example}
  Compute:
  \[
  \lim_{x\to 2}\frac{x^2-3x+2}{x-2}
  \]
  \begin{explanation}
  This limit is of the form \zeroOverZero. However, note that if we
  assume $x\ne \answer[given]{2}$, then we can write
    \[
    \frac{x^2-3x+2}{x-2} = \frac{(x-2)(\answer[given]{x-1})}{(x-2)} = \answer[given]{x-1}.
    \]
    \begin{image}
      \begin{tikzpicture}
        \begin{axis}[
            domain=-2:4,
            width=2.5in,
            axis lines =middle, xlabel=$x$, ylabel=$y$,
            every axis y label/.style={at=(current axis.above origin),anchor=south},
            every axis x label/.style={at=(current axis.right of origin),anchor=west},
            xtick={-2,...,4},
            ytick={-3,...,3},
          ]
	  \addplot [very thick, penColor, smooth] {x-1};
        \end{axis}
        \node [penColor] at (2,-.75) {$y= x-1$};
      \end{tikzpicture}
      \qquad
      \begin{tikzpicture}
	\begin{axis}[
            domain=-2:4,
            width=2.5in,
            axis lines =middle, xlabel=$x$, ylabel=$y$,
            every axis y label/.style={at=(current axis.above origin),anchor=south},
            every axis x label/.style={at=(current axis.right of origin),anchor=west},
            xtick={-2,...,4},
            ytick={-3,...,3},
          ]
	  \addplot [very thick, penColor, smooth] {x-1};
          \addplot[color=penColor,fill=background,only marks,mark=*] coordinates{(2,1)};  %% open hole
        \end{axis}
        \node [penColor] at (2,-.5) {$y=\frac{x^2-3x+2}{x-2}$};
      \end{tikzpicture}
    \end{image}
    This means that
    \[
    \lim_{x\to 2}\frac{x^2-3x+2}{x-2} = \lim_{x\to 2} (x-1).
    \]
    But now, the limit is in a form on which we can use the limit laws! 
    We have $\lim_{x\to 2} (x-1) =\answer[given]{1}$. Hence
    \[
    \lim_{x\to 2}\frac{x^2-3x+2}{x-2} = \answer[given]{1}.
    \]
  \end{explanation}
\end{example} 


Let's consider some more examples of the form \zeroOverZero.

\begin{example}
  Compute:
  \[
  \lim_{x\to1}\frac{x-1}{x^2+2x-3}.
  \]
  \begin{explanation}
    First note that
    \[
    \lim_{x\to1}\left(x-1\right)=0 \qquad\text{and}\qquad  \lim_{x\to1}\left(x^2+2x-3\right) = 0
    \]
    Hence this limit is of the form \zeroOverZero, which tells us we
    can likely cancel a factor going to $0$ out of the numerator and
    denominator.  Since $\answer[given]{(x-1)}$ is a factor going to $0$ in the
    numerator, let's see if we can factor a $\answer[given]{(x-1)}$ out of the
    denominator as well.
    \begin{align*}
      \lim_{x\to1}\frac{x-1}{x^2+2x-3}&=\lim_{x\to1}\frac{x-1}{(x-1)\answer[given]{(x+3)}} \\
      &=\lim_{x\to1}\frac{1}{\answer[given]{x+3}}\\
      &=\frac{1}{4}.
    \end{align*}
  \end{explanation}
\end{example}



\end{document}


\end{document}