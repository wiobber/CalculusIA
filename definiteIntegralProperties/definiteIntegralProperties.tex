\documentclass{ximera}

\input{../preamble.tex}



\title{Properties of the Definite Integral}

\begin{document}
\begin{abstract}
  We introduce many important properties of the definite integral.
\end{abstract}
\maketitle

\section{Properties of the Definite Integral}

The inherent relationship between definite integrals and areas gives several properties that we can use
to help us evaluate them.

\begin{theorem}[Properties of the definite integral]
Let $f$ and $g$ be defined on a closed interval $[a,b]$ that contains the
value $c$, and let $k$ be a constant. Then:
\begin{enumerate}
	\item $\int_a^a f(x)\d x = 0$
	\item $\int_a^c f(x)\d x + \int_c^b f(x)\d x = \int_a^b f(x)\d x$
	\item $\int_a^bf(x)\d x = -\int_b^a f(x)\d x$
	\item $\int_a^bk\cdot f(x)\d x = k\cdot\int_a^bf(x)\d x$
	\item $\int_a^b f(x)\pm g(x)\d x = \int_a^bf(x)\d x \pm \int_a^bg(x)\d x$

\end{enumerate}
\begin{explanation}
  We will address each property in turn:
\begin{enumerate}
\item Here, there is no ``area under the curve'' when the region has
  no width; hence this definite integral is $0$.
\item This states that total area is the sum of the areas of
  subregions. Here a picture is worth a thousand words:
  \begin{image}
    \begin{tikzpicture}[
        declare function = {f(\x) = -sin(deg(\x)) + 3;} ]
      \begin{axis}[
          domain=-.2:7, xmin =-.2,xmax=7,ymax=5,ymin=-.2,
          width=6in,
          height=3in,
          xtick={1,3.5,6}, 
          xticklabels={$a$,$c$,$b$},
          ytick style={draw=none},
          yticklabels={},
          axis lines=center, xlabel=$x$, ylabel=$y$,
          every axis y label/.style={at=(current axis.above origin),anchor=south},
          every axis x label/.style={at=(current axis.right of origin),anchor=west},
          axis on top,
          ]
        \addplot [draw=none,fill=fill4,domain=1:3.5, smooth] {f(x)} \closedcycle;
        
        \addplot [draw=none,fill=fill5,domain=3.5:6, smooth] {f(x)} \closedcycle;
        
        \addplot [very thick,penColor, smooth] {f(x)};
        \addplot [dashed] plot coordinates {(3.5,0) (3.5,{f(3.5)})};
        
        \node at (axis cs:2.25,1) {\large $\int_a^c f(x)\d x$};
        \node at (axis cs:4.75,1) {\large $\int_c^b f(x)\d x$};
      \end{axis}
    \end{tikzpicture}
  \end{image}		
  It is important to note that this still holds true even if
  $a<b<c$. We discuss this in the next point.
  
\item For now, this property can be viewed a merely a convention to
  make other properties work well. However, later we will see how this
  property has a justification all its own.

\item This states that when one scales a function by, for instance, $7$,
  the area of the enclosed region also is scaled by a factor of
  $7$.
\item This states that the integral of the sum (or difference) is the sum (or difference) of the
  integrals.
\end{enumerate}
\end{explanation}
\end{theorem}



\begin{example}
  Compute:
  \[
  \int_0^5 3-x \d x
  \]
  \begin{explanation}
    First, we graph the function $f(x)=3-x$ on the interval $[0,5]$:
  \begin{image}
  \begin{tikzpicture}
    \begin{axis}[
        width=6in,
        height=3in,
        xmin=-1, xmax=6.5,ymin=-4,ymax=4,domain=-1:6,
        axis lines =center, xlabel=$x$, ylabel=$y$,
        every axis y label/.style={at=(current axis.above origin),anchor=south},
        every axis x label/.style={at=(current axis.right of origin),anchor=west},
        axis on top,
    ] 
      \addplot [penColor,very thick,domain=0:5] {3-x};
    \end{axis}
  \end{tikzpicture}
    \end{image}
Since we see two triangles, we can cut the integral into to parts using Property (b) and use geometry to compute their areas:
    \begin{align*}
    \int_0^5 3-x \d x &= \int_0^{\answer[given]{3}} 3-x\d x + \int_{\answer[given]{3}}^5 3-x \d x\\
    &= \frac{3\cdot 3}{2} + (-1)\frac{2\cdot 2}{2}\\
    &=\answer[given]{5/2}.
    \end{align*}
  \end{explanation}
\end{example}

\begin{example}
  Compute:
  \[
  \int_0^4 2+\sqrt{16-x^2} \d x
  \]
  \begin{explanation}
    First, we split the interval into two parts using Property (e): 
    
    \[  \int_0^4 2+\sqrt{16-x^2} \d x=\int_0^4 2\d x+\int_0^4 \sqrt{16-x^2} \d x   \]
    
    Next, we graph each function on $[0,4]$ and use geometry to compute the areas:
  \begin{image}
  \begin{tikzpicture}
    \begin{axis}[
        width=6in,
        height=6in,
        xmin=-1, xmax=6.5,ymin=-4,ymax=4,domain=-1:6,
        axis lines =center, xlabel=$x$, ylabel=$y$,
        every axis y label/.style={at=(current axis.above origin),anchor=south},
        every axis x label/.style={at=(current axis.right of origin),anchor=west},
        axis on top,
    ] 
      \addplot [penColor,very thick,domain=0:4] {2};
                \node at (axis cs:3,2.5) [penColor2] {\LARGE $y=2$};
    \end{axis}
  \end{tikzpicture}

    \begin{tikzpicture}
    \begin{axis}[
        width=6in,
        height=6in,
        xmin=-1, xmax=5,ymin=-1,ymax=5,domain=-1:5,
        axis lines =center, xlabel=$x$, ylabel=$y$,
        every axis y label/.style={at=(current axis.above origin),anchor=south},
        every axis x label/.style={at=(current axis.right of origin),anchor=west},
        axis on top,
    ] 
      \addplot [penColor,very thick,domain=0:4] {(16-x^2)^(1/2)};
      \node at (axis cs:2.5,4) [penColor2] {\LARGE $y=\sqrt{16-x^2}$};
    \end{axis}
  \end{tikzpicture}
    \end{image}


    \begin{align*}
  \int_0^4 2+\sqrt{16-x^2} \d x&=\int_0^4 2\d x+\int_0^4 \sqrt{16-x^2} \d x \\
    &= (2)(4) + \frac{\pi\cdot 4^2}{4}\\
    &=8+4\pi.
    \end{align*}
  \end{explanation}
\end{example}


\begin{example}
  Compute:
  \[
  \int_0^6 |x-3| \d x
  \]
  \begin{explanation}
    The absolute value makes this tricky. Perhaps the first thing to do is
    look at a graph of $y=x-3$:
  \begin{image}
  \begin{tikzpicture}
    \begin{axis}[
        width=6in,
        height=3in,
        xmin=-.5, xmax=6.5,ymin=-4,ymax=4,domain=0:6,
        axis lines =center, xlabel=$x$, ylabel=$y$,
        every axis y label/.style={at=(current axis.above origin),anchor=south},
        every axis x label/.style={at=(current axis.right of origin),anchor=west},
        axis on top,
    ] 
      \addplot [penColor,very thick,domain=0:6] {x-3};
    \end{axis}
  \end{tikzpicture}
    \end{image}
    Now we can graph $y=|x-3|$ by reflecting the negative portion over the x-axis:
    \begin{image}
  \begin{tikzpicture}
    \begin{axis}[
        width=6in,
        height=3in,
        xmin=-.5, xmax=6.5,ymin=-4,ymax=4,domain=0:6,
        axis lines =center, xlabel=$x$, ylabel=$y$,
        every axis y label/.style={at=(current axis.above origin),anchor=south},
        every axis x label/.style={at=(current axis.right of origin),anchor=west},
        axis on top,
    ] 
      \addplot [draw=none, fill=fillp,domain=0:3] {3-x} \closedcycle;
      \addplot [draw=none, fill=fillp,domain=3:6] {x-3} \closedcycle;
      \addplot [penColor,very thick,domain=3:6] {x-3};
      \addplot [penColor,very thick,domain=0:3] {3-x};
    \end{axis}
  \end{tikzpicture}
    \end{image}
    Now we see that we really have two triangles, each with base $3$
    and height $3$.  Hence
    \begin{align*}
    \int_0^6 |x-3| \d x &= \int_0^3 \answer[given]{3-x} \d x + \int_3^6 \answer[given]{x-3} \d x\\
    &= \frac{3\cdot 3}{2} + \frac{3\cdot 3}{2}\\
    &=\answer[given]{9}.
    \end{align*}
  \end{explanation}
\end{example}

\begin{definition}
  A function $f$ is an \dfn{odd} function if for all $x$
  \[
  f(-x) = -f(x),
  \]
  and a function $g$ is an \dfn{even} function if for all $x$
  \[
  g(-x) = g(x).
  \]
\end{definition}

The names \textit{odd} and \textit{even} come from the fact that these
properties are shared by functions of the form $x^n$ where $n$ is
an odd or even integer. For example, if $f(x) = x^3$, then
\[
f(-7) = -f(7),
\]
and if $g(x) = x^4$, then
\[
g(-7) = g(7).
\]
Geometrically, even functions have \textit{horizontal
  symmetry}, which is symmetry about the y-axis. Cosine is an even function:
\begin{image}
 \begin{tikzpicture}
	\begin{axis}[
            xmin=-6.75,xmax=6.75,ymin=-1.5,ymax=1.5,
            axis lines=center,
            xtick={-6.28, -4.71, -3.14, -1.57, 0, 1.57, 3.142, 4.71, 6.28},
            xticklabels={$-2\pi$,$-3\pi/2$,$-\pi$, $-\pi/2$, $0$, $\pi/2$, $\pi$, $3\pi/2$, $2\pi$},
            ytick={-1,1},
            %ticks=none,
            width=6in,
            height=3in,
            unit vector ratio*=1 1 1,
            xlabel=$\theta$, ylabel=$x$,
            every axis y label/.style={at=(current axis.above origin),anchor=south},
            every axis x label/.style={at=(current axis.right of origin),anchor=west},
          ]        
          \addplot [very thick, penColor2, samples=100,smooth, domain=(-6.75:6.75)] {cos(deg(x))};
          \node at (axis cs:-2,.75) [penColor2] {$x=\cos(\theta)$};
        \end{axis}
\end{tikzpicture}
\end{image}
On the other hand, odd functions have $180^\circ$ \textit{rotational symmetry}, which is symmetry about the origin. Sine is an odd function:
\begin{image}
\begin{tikzpicture}
	\begin{axis}[
            xmin=-6.75,xmax=6.75,ymin=-1.5,ymax=1.5,
            axis lines=center,
            xtick={-6.28, -4.71, -3.14, -1.57, 0, 1.57, 3.142, 4.71, 6.28},
            xticklabels={$-2\pi$,$-3\pi/2$,$-\pi$, $-\pi/2$, $0$, $\pi/2$, $\pi$, $3\pi/2$, $2\pi$},
            ytick={-1,1},
            %ticks=none,
            width=6in,
            height=3in,
            unit vector ratio*=1 1 1,
            xlabel=$\theta$, ylabel=$x$,
            every axis y label/.style={at=(current axis.above origin),anchor=south},
            every axis x label/.style={at=(current axis.right of origin),anchor=west},
          ]        
          \addplot [very thick, penColor, samples=100,smooth, domain=(-6.75:6.75)] {sin(deg(x))};
          
          \node at (axis cs:3.3,.75) [penColor] {$x=\sin(\theta)$};
        \end{axis}
\end{tikzpicture}
\end{image}
\begin{question}
  Let $f$ be an odd function defined for all real numbers. Compute:
  \[
  \int_{-2}^2 f(x) \d x \begin{prompt}=\answer{0}\end{prompt}
  \]
  \begin{hint}
    Since our function is odd, it must look something like:
    \begin{image}
      \begin{tikzpicture}
        \begin{axis}[
            xmin=-2.5, xmax=2.5,ymin=-1,ymax=1,domain=-2.2:2.2,
            axis lines =center, xlabel=$x$, ylabel=$y$,
            every axis y label/.style={at=(current axis.above origin),anchor=south},
            every axis x label/.style={at=(current axis.right of origin),anchor=west},
            axis on top,
          ] 
          \addplot [penColor,very thick,smooth] {sin(deg(x))*sin(deg(x^2/1.3))};
        \end{axis}
      \end{tikzpicture}
    \end{image}
  \end{hint}
  \begin{hint}
    The integral above computes the following (signed) area:
    \begin{image}
      \begin{tikzpicture}
        \begin{axis}[
            xmin=-2.5, xmax=2.5,ymin=-1,ymax=1,domain=-2.2:2.2,
            axis lines =center, xlabel=$x$, ylabel=$y$,
            every axis y label/.style={at=(current axis.above origin),anchor=south},
            every axis x label/.style={at=(current axis.right of origin),anchor=west},
            axis on top,
          ]
          \addplot [draw=none,fill=fillp,domain=0:2, smooth] {sin(deg(x))*sin(deg(x^2/1.3))} \closedcycle;
          \addplot [draw=none,fill=filln,domain=-2:0, smooth] {sin(deg(x))*sin(deg(x^2/1.3))} \closedcycle;
          \addplot [penColor,very thick,smooth] {sin(deg(x))*sin(deg(x^2/1.3))};
        \end{axis}
      \end{tikzpicture}
    \end{image}
  \end{hint}
\end{question}


\begin{question}
  Let $f$ be an odd function defined for all real numbers. Which of
  the following are equal to
  \[
  \int_2^4 f(x) \d x ?
  \]
  \begin{selectAll}
    \choice{$\int_{4}^{2} f(x) \d x$}
    \choice{$\int_{-4}^{-2} f(x) \d x$}
    \choice[correct]{$\int_{-2}^{-4} f(x) \d x$}
    \choice[correct]{$\int_{-2}^{4} f(x) \d x$}
    \choice{$\int_{4}^{-2} f(x) \d x$}
    \choice[correct]{$\int_{2}^{-4} f(x) \d x$}
    \choice{$\int_{-4}^{2} f(x) \d x$}
    \choice[correct]{$-\int_{-4}^{2} f(x) \d x$}
    \choice[correct]{$-\int_{-4}^{-2} f(x) \d x$}
  \end{selectAll}
\end{question}




\section{Signed verses geometric area}


We know that the signed area between a curve $y=f(x)$ and the $x$-axis
on $[a,b]$ is given by
\[
\int_a^b f(x) \d x.
\]
On the other hand, if we want to know the \textit{geometric area},
meaning the total area, we compute
\[
\int_a^b |f(x)| \d x.
\]
\begin{question}
  True or false:
  \[
  \int_a^b |f(x)| \d x = \left|\int_a^b f(x) \d x\right|
  \]
  \begin{multipleChoice}
    \choice{true}
    \choice[correct]{false}
  \end{multipleChoice}
  \begin{feedback}
    Consider $f(x) = x^3$ on the interval $[-1,1]$. Here
    \[
    \int_a^b |f(x)| \d x = 1/2 \qquad\text{but}\qquad \left|\int_a^b
    f(x) \d x\right| = 0.
    \]
  \end{feedback}
\end{question}


%THIS GOT MOVED TO ANOTHER SECTION
%Let's finish by looking at a `backwards' problem.
%
%\begin{example}
%  Compute this limit:
%  \[
%  \lim_{n\to \infty} \sum_{k=1}^n \left(\sqrt{1-\left(-1+\frac{2k}{n}\right)^2}\right)
%  \left(\frac{2}{n}\right)
%  \]
%  \begin{explanation}
%    This is a limit of Riemann sums!  Specifically, it is a limit of
%    Riemann sums of $n$ rectangles, where
%    \[
%    \Delta x = \answer[given]{\frac{2}{n}}
%    \]
%    and
%    \[
%    x_k^* = -1+\frac{2k}{n}.
%    \]
%    Hence, we may rewrite this as
%    \[
%    \lim_{n\to \infty} \sum_{k=1}^n \left(\sqrt{1-(x_k^*)^2}\right)
%    \Delta x.
%    \]
%    Now we see that this computes the area between the $x$-axis and
%    the curve $y = \sqrt{1-x^2}$. Let's see it:
%    \begin{image}
%  \begin{tikzpicture}
%    \begin{axis}[
%        width=6in,
%        %height=3in,
%        unit vector ratio*=1 1 1,            
%        xmin=-1.1, xmax=1.1,ymin=-.1,ymax=1.1,
%        axis lines =center, xlabel=$x$, ylabel=$y$,
%        every axis y label/.style={at=(current axis.above origin),anchor=south},
%        every axis x label/.style={at=(current axis.right of origin),anchor=west},
%        axis on top,
%    ] 
%      \addplot [draw=none, fill=fillp,samples=200,domain=-1:1] {sqrt(1-x^2)} \closedcycle;
%      
%      \addplot [penColor,very thick,samples=200,domain=-1:1] {sqrt(1-x^2)};
%    \end{axis}
%  \end{tikzpicture}
%    \end{image}
%    By geometry, we know that this semicircle has area $\answer[given]{\pi/2}$. Hence
%    \[
%    \lim_{n\to \infty} \sum_{k=1}^n \left(\sqrt{1-(x_k^*)^2}\right)\Delta x =\answer[given]{\pi/2}.
%    \]
%  \end{explanation}
%\end{example}












\subsection{Learning Objectives}
After completing this section, students should be able to:
\vspace{.05in}

\noindent$\bullet$ Compute definite integrals using geometry.
\\$\bullet$ Compute definite integrals using the properties of integrals.
\\$\bullet$ Justify the properties of definite integrals using algebra or geometry.
\\$\bullet$ Split the area under a curve into several pieces to aid with calculations.
\\$\bullet$ Use symmetry to calculate definite integrals.
\\$\bullet$ Explain geometrically why symmetry of a function simplifies calculation of some definite integrals.
\\$\bullet$ Solve ``backwards" limit problems by interpreting the limit as a definite integral.

\end{document}