\documentclass{ximera}

\input{../preamble.tex}


\title{2.11 - Derivatives as Functions and Graphs}

\begin{document}
\begin{abstract}
\end{abstract}
\maketitle

Previously, we have found the derivative of a function at a specific value. Knowing that the derivative can mean something, for example, instantaneous velocity, what if we wanted to know the instantaneous velocity at several different times? Do we need to do the calculation separately for each time we want to know about?

\section{The derivative of a function, as a function}

First, we have to find an alternate definition for $f'(a)$, the derivative of a function $f$  at $a$.

 Let's start with
the average rate of change of the function $f$ as the input changes from $a$ to $x$. We will introduce a new variable, $h$, to denote the difference between $x$ and $a$. That
is $x-a=h$ or $x=a+h$. Take a look at the figure below.
 \begin{image}
\begin{tikzpicture}
  \begin{axis}[
      domain=0:3, range=0:6,ymax=6,ymin=0,
      axis lines =left, xlabel=$x$, ylabel=$y$,
      every axis y label/.style={at=(current axis.above origin),anchor=south},
      every axis x label/.style={at=(current axis.right of origin),anchor=west},
            xtick={1,1.666}, ytick={1,3},
            xticklabels={$a$,$a+h$}, yticklabels={$f(a)$,$f(a+h)$},
            axis on top,
    ]         
         % \addplot [penColor2!15!background, domain=(0:2)] {-3.348+4.348*x};
        %  \addplot [penColor2!32!background, domain=(0:2)] {-2.704+3.704*x};
        %  \addplot [very thick,penColor2, domain=(0:2)] {-1.994+2.994*x};         
        %  \addplot [penColor2!66!background, domain=(0:2)] {-1.326+2.326*x}; 
         % \addplot [penColor2!83!background, domain=(0:2)] {-0.666+1.666*x};
	  \addplot [textColor,dashed] plot coordinates {(1,0) (1,1)};
          \addplot [textColor,dashed] plot coordinates {(0,1) (1,1)};
          \addplot [textColor,dashed] plot coordinates {(0,3) (1.666,3)};
          \addplot [textColor,dashed] plot coordinates {(1.666,0) (1.666,3)};
          \addplot [very thick,penColor, smooth,domain=(0:1.833)] {-1/(x-2)};
            \addplot[decoration={brace,mirror,raise=.06cm},decorate,thin] plot coordinates
                       {(1,0.95) (1.666,0.95)};
                         \addplot[decoration={brace,mirror,raise=.06cm},decorate,thin] plot coordinates
                       {(1.67,1) (1.67,3)};

          \addplot[color=penColor,fill=penColor,only marks,mark=*] coordinates{(1.666,3)};  %% closed hole          
          \addplot[color=penColor,fill=penColor,only marks,mark=*] coordinates{(1,1)};  %% closed hole          
         % \addplot [very thick,penColor2, smooth,domain=(0:2)] {x};
          \addplot [very thick,penColor2,->]  plot coordinates {(1.666,1) (1.666,3)};
             \addplot [very thick,penColor4,->]  plot coordinates {(1,1) (1.666,1)};
              \node at (axis cs:0,-0.1) {$0$};
               \node at (axis cs:1.35,0.5) {$h$ };
    
                \node at (axis cs:1.77,1.9) {$\Delta y$ };
                      \node at (axis cs:1.2,5) {$=\frac{\Delta y}{h}=\frac{f(a+h)-f(a)}{h}$ };
                       \node at (axis cs:0.4,5.15) {average rate};
                       \node at (axis cs:0.447,4.84){of change};
                        \node[color=penColor] at (axis cs:1.794,3.8){\large$f$};
        \end{axis}
\end{tikzpicture}
\end{image}
Now we can write
\[
{\text{average rate of change  }}=\frac{f(a+h)-f(a)}{(a+h)-a}=\frac{f(a+h)-f(a)}{h}
  \]
  What happens if $h\to0$? In other words, what is the meaning of the limit
\[
 \lim_{h\to 0} \frac{f(a+h)-f(a)}{h}?
\]
Obviously, this limit represents $f'(a)$, the instantaneous rate of change of $f$ at $a$! Therefore, 
we have 
an alternate way of writing the definition of the derivative at the point $a$, namely
\[
f'(a) = \lim_{h\to 0} \frac{f(a+h)-f(a)}{h}.
\]

An easy way to avoid doing too much work is to replace the number you want to find the derivative at with a variable, $x$. Then, you can simply plug whatever time you like into the result after you've evaluated the limit:

  \begin{image}
    \begin{tikzpicture}
      \node at (0,0) {
      $ \underbrace{\begin{aligned}
          f'(3) &= \lim_{h\to 0}\frac{f(3+h)-f(3)}{h}\\
          &= \lim_{h\to 0}\frac{(3+h)^2-9}{h}\\
          &= \lim_{h\to 0}\frac{9+6h+h^2-9}{h}\\
          &= \lim_{h\to 0}\frac{6h+h^2}{h}\\
          &= \lim_{h\to 0}(6+h)\\
          &= 6.
        \end{aligned}}_{\text{plugging in}}$};
      \node at (5,0) {
      $ \underbrace{\begin{aligned}
            f'(x) &= \lim_{h\to 0}\frac{f(x+h)-f(x)}{h}\\
            &= \lim_{h\to 0}\frac{(x+h)^2-x^2}{h}\\
            &= \lim_{h\to 0}\frac{x^2+2xh+h^2-x^2}{h}\\
            &= \lim_{h\to 0}\frac{2xh+h^2}{h}\\
            &= \lim_{h\to 0}(2x+h)\\
            &= 2x,\\
            \text{so }f'(3) &=6.
        \end{aligned}}_{\text{working with $x$}}$};
    \end{tikzpicture}
  \end{image}

Notice that from this process, if you give me a specific value $x$, I know a rule by which I can give you back a number! Recall our definition of a function - this rule can express a function of $x$.Do you think we can always plug whatever $x$ we like into this limit and always get an answer?


\begin{problem}
  Suppose you have a function $f$. Which of the following are true?
  \begin{selectAll}
    \choice{The domain of $f'$ is equal to the domain of $f$.}
    \choice{The range of $f'$ is equal to the range of $f$.}
    \choice[correct]{The domain of $f'$ is a subset of the real numbers.}
    \choice[correct]{The range of $f'$ is a subset of the real numbers.}
    \choice{The domain of $f'$ is functions from the real numbers to
      the real numbers.}
      \choice{The range of $f'$ is functions from the real numbers to
      the real numbers.}
  \end{selectAll}
\end{problem}

\begin{problem}
Find $g'(2)$ for $g(x) = x^2 + 1$ using both methods described above.
\begin{prompt}
\[
g'(2) = \answer{4}
\]
\end{prompt}
\end{problem}

\begin{example}
 Let $f(x) = x^2-2x$. Using the alternate expression for the derivative, find the slope of the tangent line to the curve $y=f(x)$ at the point $(2,f(2))$.
  \begin{explanation}
   The slope of the tangent line is given by the derivative, $f'(2)$.
    \[
    f'(2) =  \lim_{h\to 0}\frac{f\left(\answer[given]{2+h}\right)-f\left(2\right)}{h}.
    \]
    Now substitute in for the function we know,
    \[
    f'(2) = \lim_{h\to 0}\frac{(2+h)^2-2(2+h) -0}{h}.
    \]
    Now expand the numerator of the fraction,
    \[
     f'(2) =\lim_{h\to 0}  \frac{4+4h+h^2-4-2h }{h}.
    \]
    Now combine like-terms,
    \[
    f'(2) = \lim_{h\to 0} \frac{2h+h^2}{h}.
    \]
    Factor an $h$ from every term in the numerator,
    \[
   f'(2) =  \lim_{h\to 0}\frac{\cancel{h}\left(2+h\right)}{\cancel{h}}.
    \]
  Compute the limit,
    \[
     f'(2) =  \lim_{h\to 0}(2+h)=\answer[given]{2}. 
    \]
  \end{explanation}
\end{example}


	
This alternate definition of the derivative of $f$ at $a$, namely,


\[
f'(a) = \lim_{h\to 0}\frac{f(a+h)-f(a)}{h},
\]
(provided that the limit exists), allows us to define $f'(x)$ 
 for any value of $x$, 

\[
f'(x) = \lim_{h\to 0}\frac{f(x+h)-f(x)}{h},
\]
(provided that the limit exists).

And this is how we define a new function, $f'$, the derivative of $f$. The domain of $f'$ consists of all points in the domain of $f$ where the function $f$ is differentiable.
$f'(x)$ gives us the instantaneous rate of change of $f$ at any point $x$ in the domain of $f'$.\\
\begin{comment}
\begin{warning}
  The notation:
  \begin{quote}
  $f'(a)$ means take the derivative of $f$ first, then evaluate at
    $x=a$.
  \end{quote}
  In other words, given $f$ a function of $x$
  \[
  f'(a) = \eval{\ddx f(x)}_{x=a}.
  \]
\end{warning}
\end{comment}
Given a function $f$ from  some set of real numbers to the real numbers, the
derivative $f'$ is also a function from some set of real numbers to the real
numbers. Understanding the relationship between the \textit{functions}
$f$ and $f'$ helps us understand any situation (real or imagined)
involving changing values.
\begin{comment}
\begin{question}
  Let $f(x) = 3x+2$. What is $f'(-1)$?
  \begin{multipleChoice}
    \choice{$f'(-1) = 0$ because $f'(3)$ is a number, and a number corresponds to a horizontal line, which has a slope of zero.}
    \choice[correct]{$f'(-1) = 3$ because $y=f(x)$ is a line with slope $3$.}
    \choice{We cannot solve this problem yet.}
  \end{multipleChoice}
\end{question}
\end{comment}
\begin{example}
	Given the function $f(x) = 3x+2$, find  $f'(x)$. \\
	
	\begin{explanation}
		Start with the definition of $f'(x)$
		\[
		f'(x) = \lim_{h\to0}\frac{f(x+h)-f(\answer[given]{x})}{h}.
		\]
		Replace $f$ with its formula,
		\[
		f'(x) = \lim_{h\to0}\frac{3(x+h)+2-\left(\answer[given]{3x+2}\right)}{h}.
		\]
		Simplify the top,
		\[
		f'(x) = \lim_{h\to0}\frac{3\cancel{h}}{\cancel{h}}.
		\]
		Evaluate the limit.		
		\[
		f'(x) = \lim_{h\to0}{3}=\answer[given]{3}.
		\]

		
	\end{explanation}
\end{example}

\section{Does the derivative always exist?}

\begin{example}
	Given the function $f(x) = |x|$, find  $f'(x)$. \\
	
	\begin{explanation}
	Recall, the  domain of $f$ is  $\RR$ and 
  $f$ is in fact a piecewise defined function, since
  \[
f(x) = |x|=
\begin{cases}
  \answer[given]{-x} &\text{if $x<0$,}\\
  \answer[given]{x} &\text{if $x\ge 0$}.
\end{cases}
\]
We will first compute $f'(x)$ when $x>0$.
		Start with the definition of $f'(x)$
		\[
		f'(x) = \lim_{h\to0}\frac{f(x+h)-f(\answer[given]{x})}{h}.
		\]
		Replace $f$ with its formula,
		\[
		f'(x) = \lim_{h\to0}\frac{|x+h|-\answer[given]{|x|}}{h}.
		\]
		\[
		f'(x) = \lim_{h\to0}\frac{x+h-\answer[given]{x}}{h}.
		\]
		Note: When $x>0$, then for all small enough values of  $h$ it follows that $x+h>0$. 
		Therefore, $|x+h|=x+h$.
		Now we have
		\[
		f'(x) = \lim_{h\to0}\frac{h}{h}= \lim_{h\to0}\answer[given]{1}=\answer[given]{1}.
		\]
	Now, we can compute the derivative 	$f'(x)$ when $x<0$.
\[
		f'(x) = \lim_{h\to0}\frac{f(x+h)-f(\answer[given]{x})}{h}.
		\]
		Replace $f$ with its formula,
		\[
		f'(x) = \lim_{h\to0}\frac{|x+h|-\answer[given]{|x|}}{h}.
		\]
		\[
		f'(x) = \lim_{h\to0}\frac{-x-h-\answer[given]{-x}}{h}.
		\]
		Note: When $x<0$, then for all small enough values of  $h$ it follows that $x+h<0$. 
		Therefore, $|x+h|=-(x+h)=-x-h$.
		Now we have
		\[
		f'(x) = \lim_{h\to0}\frac{-h}{h}= \lim_{h\to0}\answer[given]{-1}=\answer[given]{-1}.
		\]
		What remains to be done is to check whether the derivative $f'(0)$ exists.
		Start with the definition of $f'(0)$
		\[
		f'(0) = \lim_{h\to0}\frac{f(0+h)-f(\answer[given]{0})}{h}.
		\]
		Replace $f$ with its formula,
		\[
		f'(0) = \lim_{h\to0}\frac{|0+h|-\answer[given]{|0|}}{h}.
		\]
		\[
		f'(0) = \lim_{h\to0}\frac{|h|}{h}.
		\]
		Note: When $h>0$, then $|h|=h$, but when $h<0$, then $|h|=-h$. 
		Therefore, instead of computing the limit above, we will compute the two one-sided limits and compare them.
		
		\[
		 \lim_{h\to0^+}\frac{|h|}{h}= \lim_{h\to0^+}\frac{h}{h}=\lim_{h\to0^+}\answer[given]{1}=\answer[given]{1};
		\]
		\[
		 \lim_{h\to0^-}\frac{|h|}{h}= \lim_{h\to0^-}\frac{-h}{h}=\lim_{h\to0^-}\answer[given]{-1}=\answer[given]{-1};
		\]
		Since the two one-sided limits are not equal it follows that 
		\[
		 \lim_{h\to0}\frac{|h|}{h} {\text{           DOES NOT EXIST!}}
		\]
		Therefore,   $f'(0)$ DOES NOT EXIST, which means that $f$ is NOT DIFFERENTIABLE at $x=0$!
		To summarize
		
		 \[
f'(x) =
\begin{cases}
  \answer[given]{-1} &\text{if $x<0$,}\\
  \answer[given]{1} &\text{if $x> 0$}.
\end{cases}
\]
	\end{explanation}
\end{example}

\begin{question}
  Is it true that for any function $f$ the domain of $f'$ is equal to the domain of $f$?
  \begin{prompt}
  \begin{multipleChoice}
    \choice{yes}
    \choice[correct]{no}
  \end{multipleChoice}
  \begin{feedback}
  This is not true. 
  Consider the function $f(x)=|x|$. 
The domain of $f$ is $\RR$ and the domain of $f'$ is $(-\infty,0)\cup(0,\infty)$


This example demonstrates that a function $f$ and its derivative, $f'$, may have different domains.
  \end{feedback}
  \end{prompt}
\end{question}

\begin{question}
  Can two different functions, say, $f$ and $g$, have the same derivative?
  \begin{prompt}
  \begin{multipleChoice}
    \choice[correct]{yes}
    \choice{no}
  \end{multipleChoice}
  \begin{feedback}
    Many different functions can share the same derivatives.
    Consider two different functions, $f$ and $g$, defined by\\
    $f(x)=x$ and $g(x)=x+5$.
    Then, $f'(x)=1$, and $g'(x)=1$, for all real numbers $x$.\\
    So,  the derivatives of these two different functions are equal.

  \end{feedback}
  \end{prompt}
\end{question}


\begin{comment} 
\begin{question} 
    Which of the following computes the derivative, $f'(a)$?
    \begin{selectAll}
      \choice{$\lim_{h\to 0}\frac{(f(a)+h) - f(a)}{(a+h)-a}$}
      \choice[correct]{$\lim_{h\to 0}\frac{f(a+h) - f(a)}{(a+h)-a}$}
      \choice{$\lim_{h\to 0}\frac{(f(a)-h) - f(a)}{(a-h)-a}$}
      \choice[correct]{$\lim_{h\to 0}\frac{f(a-h) - f(a)}{(a-h)-a}$}
      \choice{$\lim_{h\to 0}\frac{f(a) - (f(a)+h)}{a-(a+h)}$}
      \choice[correct]{$\lim_{h\to 0}\frac{f(a) - f(a+h)}{a-(a+h)}$}
      \choice{$\lim_{h\to 0}\frac{f(a) - (f(a)-h)}{a-(a-h)}$}
      \choice[correct]{$\lim_{h\to 0}\frac{f(a) - f(a-h)}{a-(a-h)}$}
    \end{selectAll}
\end{question}	
\end{comment} 

\section{Continuity and Differentiability}

There are connections between continuity and differentiability.

\begin{theorem}[Differentiability Implies Continuity]\index{differentiability implies continuity}
If $f$ is a differentiable function at $x = a$, then $f$ is continuous
at $x=a$.
\begin{explanation}
To explain why this is true, we are going to use the following
definition of the derivative
\[
f'(a) = \lim_{x\to a} \frac{f(x)-f(a)}{x-a}.
\]

  Assuming that $f'(a)$ exists, we want to show that $f(x)$ is
continuous at $x=a$, hence we must show that
\[
\lim_{x\to a} f(x) = f(a).
\]
Starting with
\[
\lim_{x\to a} \left(f(x) - f(a)\right)
\]
we multiply and divide by $(x-a)$ to get
\begin{align*}
  &= \lim_{x\to a} \left((x-a)\frac{f(x) - f(a)}{x-a}\right) \\
  &= \left(\lim_{x\to a} (x-a) \right) \left(\lim_{x\to a}\frac{f(x) - f(a)}{x-a}\right) &\text{Limit Law.} \\
  &= \answer[given]{0}\cdot f'(a) = \answer[given]{0}.
\end{align*}
Since 
\[
\lim_{x\to a}\left(f(x) - f(a)\right) = 0 ,
\]
we apply the Difference Law to the left hand side
\[
\lim_{x\to a}f(x) - \lim_{x\to a}f(a) = 0 ,
\]
and use continuity of a constant to obtain that
\[
\lim_{x\to a}f(x) - f(a) = 0 .
\]
Next, we add $f(a)$ on both sides and get that
\[
\lim_{x\to a}f(x) = f(a).
\]
Now we see that $\lim_{x\to a} f(x) = f(a)$, and so $f$ is continuous at
$x=a$.
\end{explanation}
\end{theorem}

This theorem is often written as its contrapositive:
\begin{quote}
If $f(x)$ is not continuous at $x=a$, then $f(x)$ is not
differentiable at $x=a$.
\end{quote}


Thus from the theorem above, we see that all differentiable functions
on $\RR$ are continuous on $\RR$. Nevertheless there are continuous
functions on $\RR$ that are not differentiable on $\RR$.

\begin{question}
  Which of the following functions are continuous but not
  differentiable on $\RR$?
  \begin{multipleChoice}
    \choice{$x^2$}
    \choice{$\lfloor x \rfloor$}
    \choice[correct]{$|x|$}
    \choice{$\frac{x^2}{x}$}
  \end{multipleChoice}
\end{question}

\begin{example}
  Consider
  \[
  f(x) = \begin{cases}
          x^2 &\text{if $x<3$,}\\
          mx+b &\text{if $x\ge 3$.}
         \end{cases}
  \]
  What values of $m$ and $b$ make $f$ differentiable at $x=3$?
  \begin{explanation}
    To start, we know that  $f$ must be continuous at $x=3$, since it has to be
    differentiable there. We will start by making $f$  continuous at
    $x=3$. Write with me:
    \begin{align*}
      \lim_{x\to 3^-} f(x) &= \answer[given]{9}\\
      \lim_{x\to 3^+} f(x) &= \answer[given]{m 3 + b}\\
      f(3) &= \answer[given]{m 3 + b}
    \end{align*}
    So for the function to be continuous, we must have
    \[
    m\cdot 3 + b =9.
    \]
    We also must ensure that the function is differentiable at $x=3$. In other words, we have to ensure that the following limit exists

 \[
 \lim_{x\to 3}\frac{f(x)-f(3)}{x-3}.\\
\]
In order to compute this limit, we have to compute the two one-sided limits
 \[
 \lim_{x\to 3^{+}}\frac{f(x)-f(3)}{x-3}\\
\]
and
\[
 \lim_{x\to 3^{-}}\frac{f(x)-f(3)}{x-3},\\
\]
since  $f(x)$ changes expression at $x=3$.
Write with me
     \begin{align*}
        \lim_{x\to 3^{-}}\frac{f(x)-f(3)}{x-3}&= \lim_{x\to 3}\frac{x^2 -9}{x-3}\\
      &= \lim_{x\to 3^{-}}\frac{\cancel{(x-3)}(x+3)}{\cancel{x-3}}\\
      &= \lim_{x\to 3^{-}}\left(x+3\right)\\
      &=6,
    \end{align*}
    and
   \begin{align*}
        \lim_{x\to 3^{+}}\frac{f(x)-f(3)}{x-3}&= \lim_{x\to 3}\frac{mx+b -(3m+b)}{x-3}\\
          &= \lim_{x\to 3^{+}}\frac{mx-3m}{x-3}\\
      &= \lim_{x\to 3^{+}}\frac{m\cancel{(x-3)}}{\cancel{x-3}}\\
      &= \lim_{x\to 3^{+}}m\\
      &=m.
    \end{align*}
    Hence, we must have
   \[
      m=6.
  \]
    Ah! So now the equation that must be satisfied
    \begin{align*}
      9 &= m\cdot 3 + b,\\
    becomes\hspace{0.3in}  9 &= 6\cdot 3 + b.\\
    \end{align*}
   Therefore, $b=\answer[given]{-9}$. Thus setting $m=\answer[given]{6}$ and
    $b=\answer[given]{-9}$ will give us a function that is differentiable (and hence
    continuous) at $x=3$.
   
  \end{explanation}
\end{example}

Can we tell from its graph whether the function is differentiable or not at a point $a$?

\begin{question}
What does the graph of a function $f$ possibly look like when $f$ is not differentiable at $a$?
\begin{explanation}

Each of the figures A-D depicts a function that is not differentiable at $a=1$. 
 \begin{image}
    \begin{tabular}{cc}
      \begin{tikzpicture}
        \begin{axis}[
          domain=-2:2,
          xmin=-2, xmax=2,
          ymin=-2, ymax=2,
          width=2.5in,
          axis lines =middle, xlabel=$x$, ylabel=$y$,
          every axis y label/.style={at=(current axis.above origin),anchor=south},
          every axis x label/.style={at=(current axis.right of origin),anchor=west},
          ]
	  \addplot [ultra thick, penColor, smooth,domain=(-2:0.95)] {x};
	   \addplot [ultra thick, penColor, smooth,domain=(1:2)] {0.5};
	   \addplot[color=penColor,fill=penColor,only marks,mark=*] coordinates{(1,0.5)}; 
	      \addplot[color=penColor,fill=background,only marks,mark=*] coordinates{(1,1)};  %% open hole
          \node at (axis cs:-1.8, 1.4 ) [penColor,anchor=west] {\large$A$};
         
        \end{axis}
      \end{tikzpicture}
      &
      \begin{tikzpicture}
        \begin{axis}[
          domain=-2:2,
          xmin=-2, xmax=2,
          ymin=-2, ymax=2,
          width=2.5in,
          axis lines =middle, xlabel=$x$, ylabel=$y$,
          every axis y label/.style={at=(current axis.above origin),anchor=south},
          every axis x label/.style={at=(current axis.right of origin),anchor=west},
          ]
	  \addplot [ultra thick, penColor, smooth,domain=(1:2)] {2-x};
	    \addplot [ultra thick, penColor, smooth,domain=(-2:1)] {x^2};

          \node at (axis cs:-1.8, 1.4 ) [penColor,anchor=west] {\large$B$};
       
        \end{axis}
      \end{tikzpicture}\\
      \begin{tikzpicture}
        \begin{axis}[
          domain=-2:2,
          xmin=-2, xmax=2,
          ymin=-2, ymax=2,
          width=2.5in,
          axis lines =middle, xlabel=$x$, ylabel=$y$,
          every axis y label/.style={at=(current axis.above origin),anchor=south},
          every axis x label/.style={at=(current axis.right of origin),anchor=west},
          ]
	  \addplot [ ultra thick, penColor, smooth,domain=(1:2)] {(x-1)^(1/3)};
	    \addplot [ultra thick, penColor, smooth,samples=100,domain=(-2:1)] {-(1-x)^(1/3)};
	
          \node at (axis cs:-1.8, 1.4 ) [penColor,anchor=west] {\large$C$};
  
              \addplot [very thick,penColor2,dashed] plot coordinates {(1,-2) (1,2)};
        \end{axis}
      \end{tikzpicture}
      &
      \begin{tikzpicture}
        \begin{axis}[
          domain=-2:2,
          xmin=-2, xmax=2,
          ymin=-2, ymax=2,
          width=2.5in,
          axis lines =middle, xlabel=$x$, ylabel=$y$,
          every axis y label/.style={at=(current axis.above origin),anchor=south},
          every axis x label/.style={at=(current axis.right of origin),anchor=west},
          ]
	  \addplot [ultra thick, penColor, smooth,samples=100, domain=(0:0.995)] {{(1-x)*sin(deg(1/(1-x)^3))}};
	   \addplot [ultra thick, penColor, smooth, domain=(-2:0)] {{(1-x)*sin(deg(1/(1-x)^3))}};
	    \addplot [ultra thick, penColor, smooth,samples=100, domain=(1.02:2)] {{(x-1)*sin(deg(1/(x-1)^3))}};
	%  \addplot[color=penColor,fill=penColor,only marks,mark=*] coordinates{(1,0)}; 
          \node at (axis cs:-1.8, 1.4 ) [penColor,anchor=west] {\large$D$};
           % \node at (axis cs:0., 1.8 ) [penColor,anchor=west] {\large$y=f(x)$};
          
        \end{axis}
      \end{tikzpicture}
    \end{tabular}
  \end{image}
 The function in figure A is not continuous at $a$, and, therefore, it is not differentiable there.

In figures $B$--$D$ the functions are continuous at $a$, but  in each case the limit
\[
 \lim_{x\to a} \frac{f(x)-f(a)}{x-a}
\]
does not exist, for a different reason.

In figure $B$
\[
 \lim_{x\to a^{+}} \frac{f(x)-f(a)}{x-a}\ne \lim_{x\to a^{-}} \frac{f(x)-f(a)}{x-a}.
\]

 
In figure $C$
\[
 \lim_{x\to a} \frac{f(x)-f(a)}{x-a}=\infty.
\]
In figure $D$
the two one-sided limits don't exist and neither one of them is infinity.

So, if  at the point $a$ a function either has a "jump" in the graph, or a corner, or what looks like a ``vertical tangent line", or if it rapidly oscillates near $a$, then the function is not differentiable at $a$.
\end{explanation}
\end{question}




\end{document}