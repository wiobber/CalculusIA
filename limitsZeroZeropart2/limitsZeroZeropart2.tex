\documentclass{ximera}

\input{../preamble.tex}


\outcome{Calculate limits of the form zero over zero.}

\title{2.4 - Limits of the form zero over zero, part 2}

\begin{document}
\begin{abstract}
\end{abstract}
\maketitle

Continuing our investgation of  \zeroOverZero\  type limits, we will investigate two more algebraic techniques that can help us determine the limit value. Previously, we have discussed only factoring.

\begin{example}
  Compute:
  \[
  \lim_{x\to 1} \frac{\frac{1}{x+1}-\frac{3}{x+5}}{x-1}.
  \]
\begin{explanation}
  We find the form of this limit by looking at the limits of the
  numerator and denominator separately
  \[
  \lim_{x\to 1}\left(\frac{1}{x+1}-\frac{3}{x+5}\right)=0\qquad\text{and}\qquad\lim_{x\to 1}\left(x-1\right)=0.
  \]
  Our limit is therefore of the form \zeroOverZero\ and we can
  probably factor a term going to $0$ out of both the numerator and
  denominator.
  \[
  \lim_{x\to 1} \frac{\frac{1}{x+1}-\frac{3}{x+5}}{x-1}
  \]
  When looking at the denominator, we hope that this
  term is $(x-1)$.  Unfortunately, it is not immediately obvious how to
  factor an $(x-1)$ out of the numerator.  In order to simplify the
  numerator, we will ``clear denominators.'' by multiplying by
  \[
  1 = \frac{(x+1)(x+5)}{(x+1)(x+5)}
  \]
  this will allow us to cancel immediately
\begin{align*}
  \lim_{x\to 1}& \frac{\frac{1}{x+1}-\frac{3}{x+5}}{x-1}  \cdot \frac{(x+1)(x+5)}{(x+1)(x+5)} \\
  &= \lim_{x\to 1}\frac{(x+5)-3(x+1)}{(x+1)(x+5)(x-1)}.
\end{align*}

Now we will multiply out the numerator.  Note that we do not want to
multiply out the denominator because we already have an $(x-1)$
factored out of the denominator and that was the goal.

\[
\lim_{x\to 1}\frac{(x+5)-3(x+1)}{(x+1)(x+5)(x-1)}
\]
\begin{align*}
  &= \lim_{x\to 1}\frac{x+5-3x-3}{(x+1)(x+5)(x-1)} \\
  &= \lim_{x\to 1}\frac{-2x+2}{(x+1)(x+5)(x-1)}\\
  &= \lim_{x\to 1}\frac{-2\cancel{(x-1)}}{(x+1)(x+5)\cancel{(x-1)}}\\
  &= \lim_{x\to 1}\frac{-2}{(x+1)(x+5)}.
\end{align*}
  
We now have canceled, and can apply the usual Limit Laws.  Hence
\begin{align*}
\lim_{x\to 1} \frac{\frac{1}{x+1}-\frac{3}{x+5}}{x-1}&=\lim_{x\to
  1}\frac{-2}{(x+1)(x+5)}\\
&= \frac{-2}{((1)+1)((1)+5)} \\
&=\answer[given]{\frac{-1}{6}}.
\end{align*}
\end{explanation}
\end{example}

Finally, we'll look at one more example.

\begin{example}
  Compute:
  \[
  \lim_{x\to-1} \frac{\sqrt{x+5}-2}{x+1}.
  \]

\begin{explanation} 
  Note that 
  \[
  \lim_{x\to-1} \left(\sqrt{x+5}-2\right)=0\qquad\text{and}\qquad\lim_{x\to -1} \left(x+1\right) =0.
  \]
  Our limit is therefore of the form \zeroOverZero\ and we
  can probably factor a term going to $0$ out of both the numerator
  and denominator.  We suspect from looking at the denominator that
  this term is $(x+1)$.  Unfortunately, it is not immediately obvious
  how to factor an $(x+1)$ out of the numerator.
 
  We will use an algebraic technique called \dfn{multiplying by the
    conjugate}.  This technique is useful when you are trying to
  simplify an expression that looks like
  \[
  \sqrt{\text{something}} \pm \text{something else}.
  \]
  It takes advantage of the difference of squares rule 
  \[
  a^2-b^2=(a-b)(a+b).
  \]
  In our case, we will use $a=\sqrt{x+5}$ and $b=2$.  Write
  \[
  \lim_{x\to-1} \frac{\sqrt{x+5}-2}{x+1}
  \]
  \begin{align*}
    &= \lim_{x\to-1} \frac{\left(\sqrt{x+5}-2\right)}{(x+1)} \cdot \frac{\left(\sqrt{x+5}+2\right)}{\left(\sqrt{x+5}+2\right)} \\
&=\lim_{x\to-1} \frac{\answer[given]{\left(\sqrt{x+5}\right)^2-2^2}}{(x+1)\left(\sqrt{x+5}+2\right)} \\
&=\lim_{x\to-1} \frac{x+5-4}{(x+1)\left(\sqrt{x+5}+2\right)} \\
&=\lim_{x\to-1} \frac{\cancel{(x+1)}}{\cancel{(x+1)}\left(\sqrt{x+5}+2\right)} \\
&=\lim_{x\to-1} \frac{1}{\sqrt{x+5}+2}\\
&= \frac{1}{\sqrt{-1+5}+2}\\
&=\answer[given]{\frac{1}{4}}.
\end{align*}
\end{explanation}
\end{example}

All of the examples in this section are limits of the form \zeroOverZero.
When you come across a limit of the form \zeroOverZero, you should try
to use algebraic techniques to come up with a continuous
function whose limit you can evaluate.

Notice that we solved multiple examples of limits of the form
\zeroOverZero\ and we got different answers each time.  This tells us
that just knowing that the form of the limit is \zeroOverZero\ is not enough
to compute the limit. The moral of the story is
\begin{center}
  \textbf{Limits of the form \zeroOverZero\ can take any value.}
\end{center}

\begin{definition}
A form that give us no information about the value of the limit is
called an \dfn{indeterminate form}.

A forms that give information about the value of the limit is called a
\dfn{determinate form}.
\end{definition}  

We will see more examples of indeterminate forms in the future.



\end{document}
