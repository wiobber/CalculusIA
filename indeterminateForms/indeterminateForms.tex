\documentclass{ximera}

\input{../preamble.tex}

\title{7.8 - More Indeterminate Forms}



\begin{document}
\begin{abstract}

\end{abstract}
\maketitle

To deal with  indeterminate forms, we have L'H\^{o}pital's rule.

\begin{theorem}[L'H\^{o}pital's Rule]\index{L'H\^opital's Rule} 
Let $f(x)$ and $g(x)$ be functions that are differentiable near $a$.  If
\[
\lim_{x \to a} f(x) = \lim_{x \to a}g(x) = 0 \qquad \text{or} \pm \infty,
\]
and $\lim_{x \to a} \frac{f'(x)}{g'(x)}$ exists, and $g'(x) \neq 0$
for all $x$ near $a$, then 
\[
\lim_{x \to a} \frac{f(x)}{g(x)} = \lim_{x \to a} \frac{f'(x)}{g'(x)}.
\]
\end{theorem}



For the other indeterminate forms, L'H\^opital's Rule does not apply.  Our approach will be to modify the form so we can apply L'H\^opital's Rule.

\section{Indeterminate forms involving multiplication}
$\zeroTimesInfty$-forms arise from a limit of the form: $\lim_{x\to a} f(x)g(x)$.  To write $f(x)g(x)$ as a fraction, we remember
\[ f(x)g(x) = \frac{f(x)}{\frac{1}{g(x)}} = \frac{g(x)}{\frac{1}{f(x)}} \]

Let's work through an example.

\begin{example}\label{example:xlnx infty} 
Compute 
\[
\lim_{x\to 0^+} x\ln x.
\]
\begin{explanation}
This doesn't appear to be suitable for L'H\^{o}pital's Rule. As $x$
approaches zero, $\ln x$ goes to $-\infty$, so the product looks like
\[
(\text{something very small})\cdot (\text{something very large and
  negative}).
\] 
This product could be anything. A careful analysis is required.
Write
\[
x\ln x = \frac{\ln x}{\frac{1}{x}} = \frac{\ln x}{x^{-1}}.
\]
Set $f(x) = \ln(x)$ and $g(x) = x^{-1}$.  Since both functions are differentiable near zero and 
\[
\lim_{x\to 0+} \ln(x) = -\infty\qquad\text{and}\qquad \lim_{x\to 0+} x^{-1} = \infty,
\]
we may apply L'H\^{o}pital's rule. Write with me
\[
f'(x) = \answer[given]{x^{-1}}
\]
and
\[
g'(x) = \answer[given]{-x^{-2}},
\]
so
\begin{align*}
  \lim_{x\to 0^+} x\ln x &= \lim_{x\to 0^+} \frac{\ln x}{x^{-1}} \\
  &= \lim_{x\to 0^+} \frac{x^{-1}}{-x^{-2}}\\
  &=\lim_{x\to 0^+} -x \\
  &= 0.
\end{align*}
One way to interpret this is that since $\lim_{x\to 0^+}x\ln x = 0$,
the function $x$ approaches zero much faster than $\ln x$ approaches
$-\infty$.
\end{explanation}
\end{example}




\section{Indeterminate forms involving subtraction}

There are two basic cases here, we'll do an example of each.

\begin{example}
Compute
\[
\lim_{x\to 0} \left(\cot(x) - \csc(x)\right).
\]
\begin{explanation}
Here we simply need to write each term as a fraction,
\begin{align*}
\lim_{x\to 0} \left(\cot(x) - \csc(x)\right) &= \lim_{x\to 0} \left(\frac{\cos(x)}{\sin(x)} - \frac{1}{\sin(x)}\right)\\
&= \lim_{x\to 0} \frac{\cos(x)-1}{\sin(x)} 
\end{align*}
Setting $f(x) = \cos(x)-1$ and $g(x)=\sin(x)$, both functions are differentiable near zero and 
\[
\lim_{x\to 0}(\cos(x)-1)=\lim_{x\to 0}\sin(x) = 0.
\]
We may now apply L'H\^{o}pital's rule. Write with me
\[
f'(x) = \answer[given]{-\sin(x)}
\]
and
\[
g'(x) = \answer[given]{\cos(x)},
\]
so
\begin{align*}
  \lim_{x\to 0} \left(\cot(x) - \csc(x)\right) &= \lim_{x\to 0} \frac{\cos(x)-1}{\sin(x)}\\
  &= \lim_{x\to 0} \frac{-\sin(x)}{\cos(x)} \\
  &=0.
\end{align*}
\end{explanation}
\end{example}


Sometimes one must be slightly more clever. 

\begin{example}
Compute
\[
\lim_{x\to\infty}\left(\sqrt{x^2+x}-x\right).
\]
\begin{explanation}
Again, this doesn't appear to be suitable for L'H\^{o}pital's Rule. A
bit of algebraic manipulation will help. Write with me
\begin{align*}
\lim_{x\to\infty}\left(\sqrt{x^2+x}-x\right) &= \lim_{x\to\infty}\left(x\left(\sqrt{1+1/x}-1\right)\right)\\
&=\lim_{x\to\infty}\frac{\sqrt{1+1/x}-1}{x^{-1}}
\end{align*}
Now set $f(x) = \sqrt{1+1/x}-1$, $g(x) = x^{-1}$. Since both
  functions are differentiable for large values of $x$ and 
\[
\lim_{x\to\infty} (\sqrt{1+1/x}-1) = \lim_{x\to\infty}x^{-1} = 0, 
\]
we may apply L'H\^{o}pital's rule. Write with me
\[
f'(x) = \answer[given]{(1/2)(1+1/x)^{-1/2}\cdot(-x^{-2})}
\]
and
\[
g'(x) = \answer[given]{-x^{-2}}
\]
so
\begin{align*}
\lim_{x\to\infty}\left(\sqrt{x^2+x}-x\right) &= \lim_{x\to\infty}\frac{\sqrt{1+1/x}-1}{x^{-1}} \\
&= \lim_{x\to\infty}\frac{(1/2)(1+1/x)^{-1/2}\cdot(-x^{-2})}{-x^{-2}} \\
&= \lim_{x\to\infty} \frac{1}{2\sqrt{1+1/x}}\\
&= \frac{1}{2}.
\end{align*}
\end{explanation}
\end{example}


\section{Exponential Indeterminate Forms}

There is a standard trick for dealing with the indeterminate forms
\[
\text{\oneToInfty},\quad \text{\zeroToZero},\quad \text{\inftyToZero}.
\]
Given $u(x)$ and $v(x)$ such that
\[
\lim_{x\to a}u(x)^{v(x)}
\]
falls into one of the categories described above, rewrite as
\[
\lim_{x\to a}e^{v(x)\ln(u(x))}
\]
and then examine the limit of the exponent
\[
\lim_{x\to a} v(x)\ln(u(x)) = \lim_{x\to a} \frac{\ln(u(x))}{v(x)^{-1}}
\]
using L'H\^{o}pital's rule.  Since these forms are all very similar, we
will only give a single example.


\begin{example}
Compute
\[
\lim_{x\to \infty}\left(1 + \frac{1}{x}\right)^x.
\]
\begin{explanation}
Write
\[
\lim_{x\to \infty}\left(1 + \frac{1}{x}\right)^x = \lim_{x\to \infty}e^{x\ln\left(1 + \frac{1}{x}\right)}.
\]
So now look at the limit of the exponent
\[
\lim_{x\to\infty} x\ln\left(1 + \frac{1}{x}\right) = \lim_{x\to\infty} \frac{\ln\left(1 + \frac{1}{x}\right)}{x^{-1}}.
\]
Setting $f(x) = \ln\left(1 + \frac{1}{x}\right)$ and $g(x) = x^{-1}$,
both functions are differentiable for large values of $x$ and
\[
\lim_{x\to \infty}\ln\left(1 + \frac{1}{x}\right)=\lim_{x\to \infty}x^{-1} = 0.
\]
We may now apply L'H\^{o}pital's rule. Write
\[
f'(x) = \answer[given]{\frac{-x^{-2}}{1 + \frac{1}{x}}}
\]
and
\[
g'(x) = \answer[given]{-x^{-2}},
\]
so
\begin{align*}
\lim_{x\to\infty} \frac{\ln\left(1 + \frac{1}{x}\right)}{x^{-1}} &= \lim_{x\to\infty} \frac{\frac{-x^{-2}}{1 + \frac{1}{x}}}{-x^{-2}} \\
&=\lim_{x\to\infty} \frac{1}{1 + \frac{1}{x}}\\
&=1.
\end{align*}
Hence, 
\[
\lim_{x\to \infty}\left(1 + \frac{1}{x}\right)^x = \lim_{x\to \infty}e^{x\ln\left(1 + \frac{1}{x}\right)} =e^{1} = e.
\]
\end{explanation}
\end{example}





\end{document}
